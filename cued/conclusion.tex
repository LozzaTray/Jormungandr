\section{Conclusion}
\label{sec:conclusion}

The Feature-First Block Model (FFBM) introduced 
in this paper is a new generative model for labelled networks,
developed to address difficulties of other graphical models 
when testing how vertex features affect community structure. The idea 
is to divide the graph into its most natural partition and determine whether 
the vertex features can accurately explain this partition. 
It is relatively easy to find vertex features that are in some way 
correlated with the graphical structure. Nonetheless, only when 
we find the feature that best describes the most pronounced partition 
do we have a stronger case for causation.

Using this new model,
we go on to describe an efficient inference algorithm to sample 
the parameters of the FFBM. 
This takes the form of a two-level Markov chain,
used to sample the block memberships $b$ and block generator 
parameters $\theta$. These chains can either be executed
in parallel or 
the empirical mean of the $b$-samples can be used
as the input to the $\theta$-chain. The latter option
reduces the variance in our evaluation of the target distribution 
and thus shortens burn-in.

The overall method is shown to be effective at extracting and describing 
the most natural communities in a labelled network. Nevertheless, the approach 
can only currently explain the structure at the macro-scale. We cannot 
explain structure within each block. Future work will benefit from extending 
the FFBM to be hierarchical in nature. That way, the structure of the network 
can be explained at all length-scales of interest. So long as data 
collection techniques remain ethical and care is taken to respect 
personal privacy, such empowered decision-making can only help humankind.


