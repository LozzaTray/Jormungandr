\section{Conclusion}
\label{sec:conclusion}

The FFBM was developed to address the shortcomings of other graphical models when testing how vertex features affect community structure. The idea is to divide the graph into its most natural partition and test whether the vertex features can accurately explain this partition. It is very easy to find vertex features that are in some way correlated with the graphical structure. Nonetheless, only when we find the feature that best describes the most pronounced partition do we have a stronger case for causation.

With the newly-defined FFBM, we go on to present an efficient inference algorithm to sample the parameters $\theta$ of the feature-to-block generator. This is introduced as two concurrent Markov chains to sample the block memberships $b$ and block generator parameters $\theta$. Nevertheless, we can serialise the chains and use the empirical mean of the $b$-samples as the input to our $\theta$-chain. This reduces the variance in our evaluation of the target distribution and thus shortens burn-in.

The overall method is shown to be effective at extracting and describing the most natural communities in a labelled network. Nevertheless, the approach can only currently explain the structure at the macro-scale. We cannot explain structure within each block. Future work will benefit from extending the FFBM to be hierarchical in nature. That way, the structure of the network can be explained at all length-scales of interest. So long as data collection techniques remain ethical and care is taken to respect personal privacy, such empowered decision-making can only help humankind.