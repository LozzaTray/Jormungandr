\section{Preliminaries}

We first need a model for community-like structure in a network. For this we adopt the widely-used stochastic block model (SBM).
Each node in the graph belongs to a unique community called a block. The probability that two nodes are connected depends only on the block memberships of each. Specifically, we will use the microcanonical variant of the SBM, proposed by \citet{Peixoto-Bayesian-Microcanonical}. To allow for degree-variability between members of the same block, we adopt the degree-corrected formulation (DC-SBM), defined in~(\ref{defn:microcan-dc-sbm}).

For each integer $K\geq 1$, we use the notation $[K]:=\{1,2,\ldots,K\}$.

\begin{definition}[Microcanonical DC-SBM]
	\label{defn:microcan-dc-sbm}
	Let $N \geq 1$ denote the number of vertices in an undirected graph. The block memberships are encoded by a vector $b \in [B]^N$,
where $B$ is the number of non-empty blocks.
	Let $e=(e_{rs})$ be the $N\times N$ symmetric matrix of edge counts 
between blocks, such that $e_{rs}$ is the number of edges from block $r$ to 
block $s$ -- or twice that number if $r=s$. 
	Let $k =(k_i)$ denote the vector of length $N$ containing the degree
sequence of the graph, with $k_i$ being the degree of vertex $i$.

The graph's adjacency matrix $A \in \{0,1\}^{N \times N}$ is generated 
by placing edges uniformly at random, but respecting the constraints 
imposed by $e$, $b$ and $k$ (hence the qualification `microcanonical').
Specifically, $A$ must satisfy the following, for all $r,s\in[B]$
and all $i\in[N]$:
	%
	\begin{equation}
		e_{rs} = \sum_{i, j \in [N]} A_{ij} 
	\one \{b_i = r\} \one \{b_j = s\} 
		\qquad 
		\textrm{and} \qquad
		k_i = \sum_{j \in [N]} A_{ij}.
		\label{eqn:sbm-constraints}
	\end{equation}
We use the following notation to indicate this distribution:
\begin{equation}
		A \sim \mbox{\rm DC-SBM}_{\rm MC} (b, e, k).
	\end{equation}
\end{definition}
