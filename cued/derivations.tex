\section{Derivations}

\subsection{Derivation of {\boldmath $p(b|X)$}}
\label{appdx:b|x}

We determine the form of $p(b| X)$ by integrating
out the parameters $\theta$. From the definitions, we have:
%
\begin{align*}
	p(b | X) &= \int p(b , \theta| X, \theta) d\theta = \int p(b | X, \theta) p(\theta | X) d\theta \\
	&=\int p(b | X, \theta) p(\theta) d\theta = \int \prod_{i \in [N] } \phi_{b_i}(x_i; \theta) p(\theta) d\theta \\
	&= \prod_{i \in [N]} \int \frac{\exp(w_{b_i}^T x_i) \prod_{j \in [B]} \Gaussian(w_j; 0, \sigma_\theta^2 I)}{\sum_{k \in [B]} \exp(w_{k}^T x_i)} dw_{1:B}.
\end{align*}
%
The key observation here is that 
the value of the integral is independent
of the value of $b_i \in [B]$ as
the integrand has the same form regardless of $b_i$. This is
because the prior is the same for each $w_j$. 
Therefore, the integral can only be a function of at most $x_i$ and $\sigma_\theta^2$,
which means that, as a function of $b$, $p(b|X)\propto 1$. As
$b$ takes values in $[B]^N$, we necessarily have:
%
\begin{equation}
	p(b | X) = \frac{1}{\big|[B]^N\big|}=B^{-N}.
\end{equation}

\subsection{Derivation of {\boldmath $\;U(\theta)$}}
\label{appdx:form-U}

Recall from~(\ref{eq:U}) in Section~\ref{s:sfb} that,
$$	\pi_\theta(\theta) \propto p(\theta | X, b) \propto p(b | X, \theta) p(\theta) \propto  \exp \left( - U(\theta) \right),
$$ 
so that $U$ can be expressed as,
$$
U(\theta) 
= - \left( \log p(b | X, \theta) + \log p(\theta) \right) + \textrm{const}.
$$
Writing,
$y_{ij} \coloneqq \one \left\{ b_i = j \right\}$ and 
$a_{ij} \coloneqq \phi_j(x_i; \theta)$, we have that,
%
\begin{equation}
	\log p(b | X, \theta) = \sum_{i \in [N]} \sum_{j \in [B]} y_{ij} \log a_{ij}  \quad \textrm{and} \quad
	\log p(\theta) = -\frac{D B}{2} \log 2\pi - \frac{1}{2 \sigma_\theta^2} 
	\|\theta \|^2,
	\label{eqn:U-constituent-terms}
\end{equation}
%
where
$\|\theta\|^2 = \sum_{i} \theta_{i}^2 = \sum_{j \in [B]} \|w_j\|^2$ 
is the Euclidean norm of the vector of parameters $\theta$.
Therefore, discarding constant terms, we 
obtain:
%
\begin{equation}
	U(\theta) = \left( \sum_{i \in [N]} \sum_{j \in [B]} y_{ij} \log \frac{1}{a_{ij}} \right)
	+ \frac{1}{2\sigma_\theta^2} \|\theta\|^2.
\end{equation}
%
This is exactly the representation~(\ref{eqn:U-form}), as claimed.

\subsection{Derivation of {\boldmath $\;\nabla U(\theta)$}}
\label{appdx:gradu}
Here we show how the gradient 
$\nabla U(\theta)$ can be computed explicitly.
Recall the expression for $U(\theta)$ in~(\ref{eqn:U-form}).
Writing $\theta$ as
$\theta = \left[w_1^T, w_2^T \dots w_B^T  \right]^T$,
in order to compute the gradient
$\nabla U(\theta)$ we need to compute
each of its components,
$\nicefrac{\partial U}{\partial w_k}$,
$1\leq k\leq B$.
To that end, we first compute,
%
\begin{align}
	\frac{\partial a_{ij}}{\partial w_k} &= \frac
	{x_i \exp(w_j^T x_i) \delta_{jk} \cdot \sum_{r \in [B]} \exp(w_r^T x_i) 
		- 
		\exp(w_j^T x_i) \cdot x_i \exp(w_k^T x_i)}
	{\left( \sum_{r \in [B]} \exp(w_r^T x_i) \right)^2} \nonumber \\
	&= x_i \left( a_{ij} \delta_{jk} - a_{ij}a_{ik} \right), 
	\label{eq:dadw}
\end{align}
%
where $\delta_{jk} \coloneqq \one \left\{ j = k \right\}$,
and we also easily find,
%
\begin{equation}
	\frac{ \partial}{\partial w_k} \|\theta\|^2 = \frac{\partial}{\partial w_k} \left( \sum_{r \in [B]} \|w_r\|^2 \right) = 2w_k.
	\label{eq:dtsdw}
\end{equation}
%
Using~(\ref{eq:dadw}) and~(\ref{eq:dtsdw}), we obtain,
\begin{align}
	\frac{\partial U}{\partial w_k} &= 
	\sum_{i \in [N]} \sum_{j \in [B]} y_{ij} 
	\left( -\frac{x_i}{a_{ij}} \left( a_{ij} \delta_{jk} - a_{ij} a_{ik} \right) \right)
	+ \frac{w_k}{\sigma_\theta^2} \nonumber \\
	&=  - \left( \sum_{i \in [N]} x_i \left( y_{ik} - a_{ik} \sum_{j \in [B]} y_{ij} \right)
	- \frac{w_k}{\sigma_\theta^2} \right) \nonumber \\
	&= - \left( \sum_{i \in [N]} \Big\{ x_i (y_{ik} - a_{ik}) \Big\} - \frac{w_k}{\sigma_\theta^2} \right).
\end{align}
%
This can be computed 
efficiently through matrix operations. The only property of $y_{ij}$ 
we have used in the derivation is the constraint $\sum_{j \in [B]} y_{ij} = 1$,
for all $i$.