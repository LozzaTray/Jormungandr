\section{Introduction}

Many real-world networks exhibit strong community structure, with most nodes often belonging to a densely connected cluster. 
Finding ways to recover the latent communities from the observed graph is an important
task in many applications. In this work, we restrict our attention to vertex-labelled networks, 
referring to the labels as {\em features}. A typical goal is to determine whether a given feature impacts graphical structure. Answering this requires a random graph model; the standard is the stochastic block model (SBM) \cite{vanilla-sbm}. Numerous variants of the SBM  have been proposed -- such as the MMSBM \cite{mixed-membership-sbm} and OSBM \cite{overlapping-sbm}. However, these do not automatically include vertex features in the graph generation process.

To analyse a labelled network using one of the simple SBM variants, a typical procedure would be to partition the graph into blocks grouped by distinct values of the feature of interest. The associated model can then be used to test for evidence of heterogeneous connectivity between the feature-grouped blocks. Nevertheless, this approach can only consider one feature set at a time and the feature-grouped blocks are often an unnatural partition of the graph.

We would instead prefer to partition the graph into its most natural blocks and then find which of the available features -- if any -- best predict the resulting partition. Thus motivated, we present a novel framework for modelling labelled networks, which we call the feature-first block model (FFBM). This is an extension of the SBM to labelled networks.