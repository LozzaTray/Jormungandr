Table~\ref{tab:results} summarises the results for each experiment. 
We see that the dimensionality reduction procedure 
brings the training and test losses closer. This indicates that 
the retained
are indeed well correlated with the underlying graphical 
partition and that the approach generalises correctly. The test loss variance is higher than the training loss variance as the test set is smaller and so more susceptible to variability in its construction.
%
The average description length per entity of the graph,
$\bar{S}_e$,
has very small variance, suggesting that
the detected communities can be found reliably (to within an arbitrary 
relabelling of blocks).

\paragraph{\textbf{Political books.}}

We choose to partition the network into $B=3$ communities as we only have this many distinct values for political affiliation.
From Figure~\ref{fig:polbooks-null} we see that all 3 blocks have a distinct political affiliation as their largest positive component.  
Furthermore, the training and test losses from Table~\ref{tab:results}  
are very similar and both are low in magnitude. This is strong evidence 
that political affiliation is a very appropriate explanatory 
variable for the overall network structure.
%
However, from Figure~\ref{fig:polbooks-accuracy} we see that block 1 has low accuracy. 
This suggests that detected block 1 is not solely composed of ``neutral" books but also 
contains some ``liberal" and ``conservative" authors. Examining 
Figure~\ref{fig:polbooks-graph}, we see the majority of paths between blocks 2 and 3 go through block 1.
Block 1 is in effect a bridge between the ``conservative'' and ``liberal'' blocks so it is unsurprising that some of these leak into block 1.

\paragraph{\textbf{Primary school.}}

We choose the number of communities $B=10$, in line with the total number of 
school classes. Only the pupils' class memberships (1A-5B) survive
the dimensionality-reduction process (Figure~\ref{fig:school-null});
gender and teacher/student status have been discarded,
meaning these are poor predictors of overall macro-structure.
%
Almost all blocks are composed of a single class. 
However, some blocks have two comparably strong classes as their predictors (e.g. blocks 2 and 5). 
Conversely, some classes are found to extend over two 
detected blocks (class 2B spans blocks 8 and 9) but we do 
not have a feature which explains the division.
%
Figure~\ref{fig:school-accuracy} shows excellent accuracy for most blocks. In fact the only blocks with low accuracy are those with a `school-class'
feature that spans two blocks, such that we cannot reliably distinguish between the two. This is more pronounced when we apply hard classification rather than cross-entropy loss. It is possible that there are unobserved features here
which explain this divide.

\paragraph{\textbf{Facebook egonet.}}

The retained features 
(Figure~\ref{fig:fb-null}) are those that best explain the high-level 
community structure. The majority of these are education related. 
Nevertheless, for $D'=10$ we only have good explanations for some of the detected blocks; several blocks in 
Figure~\ref{fig:fb-null} do not have high-magnitude components. This is further emphasised by the disparate accuracies in Figure~\ref{fig:fb-accuracy}.
%
For a high-dimensional feature-space, it is likely that a particular
feature may uniquely identify a small set of vertices; if these are all in the same block, then the classifier may overfit despite the penalty imposed by the prior. Indeed, we see in
Figure~\ref{fig:fb-null} that the feature `birthday-5' has a very high weight as it relates to block 1 – but it is unlikely that birthdays determine graphical structure.
