For the higher-dimensional datasets, we also apply the 
dimensionality reduction method 
of Section~\ref{sec:dim-reduction}.  
We then retrain the feature-block predictor using only the retained 
feature set $\Dcal'$, and report the log-loss over the training and 
test sets for the reduced classifier -- 
denoted $\bar{\Lcal}_0'$ and $\bar{\Lcal}_1'$ respectively. 

Table~\ref{tab:results} summarises the results for each experiment.
We see that the dimensionality reduction procedure 
brings the training and test losses closer together. This implies that 
the features we keep are indeed correlated with the underlying graphical 
partition and that the approach generalises correctly.
%
The average description length per entity,
$\bar{S}_e$, of the graph, 
has very small variance, suggesting that
the detected communities can be found reliably (to within an arbitrary 
relabelling of blocks).

\paragraph{\textbf{Political books.}}

We choose to partition the network into $B=3$ communities as we only have this many distinct values for political affiliation (conservative, liberal or neutral).
From Figure~\ref{fig:polbooks-null} we see that all 3 blocks have a distinct political affiliation as their largest positive component.  
Furthermore, the training and test losses from Table~\ref{tab:results}  
are very similar and both are low in magnitude. This is strong evidence 
that political affiliation is a very appropriate explanatory 
variable for the overall network structure.
%
However, from Figure~\ref{fig:polbooks-accuracy} we see that block 1 has low accuracy. 
This suggests that detected block 1 is not solely composed of ``neutral" books but also 
contains some ``liberal" and ``conservative" authors. Examining 
Figure~\ref{fig:polbooks-graph}, we see the majority of paths between blocks 2 and 3 go through block 1.
Block 1 is in effect a bridge between the ``conservative'' and ``liberal'' blocks so it is unsurprising that some books from either side leak into block 1.

\paragraph{\textbf{Primary school.}}

We choose the number of communities $B=10$, in line with the total number of 
school classes. Only the pupils' class memberships (1A-5B) survive
the dimensionality-reduction process (Figur~\ref{fig:school-null}e);
gender and teacher/student status have been discarded,
meaning these are poor predictors of overall macro-structure.
%
The vast majority of blocks are composed of a single class. 
However, some blocks have two comparably strong classes as their predictors (e.g. blocks 2 and 5). 
Conversely, some classes are found to extend over two 
detected blocks (class 2B spans blocks 8 and 9) but we do 
not have a feature which explains the division.
%
Figure~\ref{fig:school-accuracy} shows excellent accuracy for the majority of blocks. In fact the only blocks with low accuracy are those that have a school-class span two blocks such that we cannot reliably distinguish between the two. This is more pronounced when we apply hard classification rather than the soft cross-entropy loss. Perhapsthere are unobserved features which explain this divide.

\paragraph{\textbf{Facebook egonet.}}

We choose $B=10$ and $D'=10$ for this experiment. The selected features 
(Figure~\ref{fig:fb-null}) are those that best explain the high-level 
community structure. The majority of them are education related. 
Nevertheless, for $D'=10$ we only have good explanations for some of the detected blocks; several blocks in 
Figure~\ref{fig:fb-null} do not have high-magnitude components for $D'=10$. This is further emphasised by the disparate accuracies in Figure~\ref{fig:fb-accuracy}.