\section{Experiments}
\label{sec:experiments}

We apply the developed methods to a vareity of datasets chosen to span a range of edge counts $E$ and feature-space dimension $D$. 
However, we first require metrics to assess model performance. This can be split into two separate 
components: the microcanonical SBM fit (concerned with the $b$-samples) and 
the fit of the feature-to-block generator (concerned with the $\theta$-samples). 

Starting with the SBM, recall that the quantity $S(b)$ in~(\ref{eqn:dl-form}) 
can be interpreted as an ideal ``description length'' of the partition 
imposed by $b$. We define the average
description length per entity (nodes and edges) $\bar{S}_e$ to gauge the SBM fit:
%
\begin{equation}
	\bar{S}_e \coloneqq \frac{1}{(N+E) |\Tcal_b|} \sum_{t\in \Tcal_b} S \left( b^{(t)} \right).
	\label{eqn:mean-dl}
\end{equation}

Next, to assess the performance of the feature-to-block predictor, 
we partition the vertex set $[N]$ into training and test sets. On each experiment run,
a constant fraction $f$ of the available vertices go to form 
the training set $\Gcal_0$ and the remainder are held out to form the
test set $\Gcal_1$. These sets are chosen randomly.
The $b$-chain is run using the whole network but we only use vertices $v \in \Gcal_0$ to train the $\theta$-chain. Because $|\Gcal_0| \neq |\Gcal_1|$ in general, we cannot use the un-normalised log-target $U$ from~(\ref{eqn:U-form}) for benchmarking,
as the total cross-entropy loss scales with the size of each data set but 
the prior term stays constant. We therefore use the average cross-entropy loss 
over each set,
%
\begin{equation}
	\bar{\Lcal}_\star \coloneqq \frac{1}{|\Tcal_\theta|} \sum_{t \in \Tcal_\theta} \Lcal_\star^{(t)},
	\quad \textrm{where} \quad
	\Lcal_\star^{(t)} \coloneqq \frac{1}{|\Gcal_\star|} \sum_{i \in \Gcal_\star}\sum_{j \in [B]} \hat{y}_{ij} \log \frac{1}{\phi_j \left(x_i; \theta^{(t)} \right)},
	\label{eqn:cross-entropy-loss}
\end{equation}
%
where $\star \in \{0, 1\}$ indicates whether the training or test
set is being considered.

Nevertheless, the cross-entropy loss over the whole training set may be too coarse a measure of model fit. It is often the case that we have good feature-based explanations for some but not all of the detected blocks. We wish to define a new measure of fit specific to each detected block. This requires defining the set of all vertices associated with block $j$ as,
$
	\Bcal_\star(j) \coloneqq \{i \in \Gcal_\star : \hat{b}_i = j\}
$
where
$ 
	\quad \hat{b}_i \coloneqq \argmax_j \hat{y}_{ij}.
$
Recall that $\hat{y}_{ij}$ is our estimate for the block membership posterior~(\ref{eqn:y-hat}) using only information from the adjacency matrix $A$. Again, $\star \in \{0, 1\}$ toggles between training and test sets. Now $\Bcal_\star(j)$ is the set of vertices that have maximum a posteriori probability of belonging to block $j$. We choose to resolve ties consistently by picking the lower index. We now define the accuracy for block $j$ as,
%
\begin{equation}
	\eta_\star(j) \coloneqq \frac{1}{|\Bcal_\star (j)| \cdot 
	|\Tcal_\theta| } 
	\sum_{i \in \Bcal_\star (j)}  \sum_{t \in \Tcal_\theta}
	\one \left\{\hat{b}_i = \argmax_j \phi_j \left( x_i; \theta^{(t)} \right) \right\}.
	\label{eqn:accuracy}
\end{equation}
%
This is effectively testing whether the feature-to-block and the graph-to=block predictions agree in their largest component. We call this metric, $\eta_\star(j)$, the {\em block-accuracy} for block $j$. It is clearly bounded $0 \leq \eta_\star(j) \leq 1$, with an accuracy of 1 meaning perfect agreement for the vertices in detected block $j$.

For each of the experiments, we plot the collected $\theta$-samples for the features that survive the dimensionality reduction procedure. We also plot the per-block accuracy for the original-dimension classifiers. We discuss the results specific to each dataset in turn.

Table~\ref{tab:results} summarises the results for each experiment. We also apply the 
dimensionality reduction method 
of Section~\ref{sec:dim-reduction}
to the two higher dimensional datasets (the school and FB egonet). 
For this we use equation~(\ref{eqn:c-star}) with $k=1$,
in order to reduce the dimension from 
$D$ to a desired $D'$. 
We then retrain the feature-block predictor using only the retained 
feature set $\Dcal'$, and report the log-loss over the training and 
test sets for the reduced classifier -- 
denoted $\bar{\Lcal}_0'$ and $\bar{\Lcal}_1'$ respectively. 
These values are also given in Table~\ref{tab:results}.

Based on these results, some remarks are in order.
Firstly, the variance of the test loss $\bar{\Lcal}_1$ tends to be higher 
than the training loss $\bar{\Lcal}_0$. This is expected,
as the test set is smaller than the training set and hence 
more susceptible to variability in its construction. 
Secondly, it can be seen that the dimensionality reduction procedure 
brings the training and test losses closer together. This implies that 
the features we keep are indeed correlated with the underlying graphical 
partition and that the approach generalises correctly.

The average description length per entity,
$\bar{S}_e$, of the graph, 
has very small variance, suggesting that
the detected communities can be found reliably (to within an arbitrary 
relabelling of blocks). For reference, we plot an inferred partition for each 
of the graphs in Figure~\ref{fig:graphs-all}. The polbooks graph yields the cleanest separation between blocks but nonetheless the inferred partitions for the other datasets do succeed at dividing the graph into densely connected clusters.

\subsubsection{Political books}

\cite{polbooks} ($N=105, E=441, D=3$) -- network of Amazon political book sales, published close to the 2004 presidential election. Two books are connected if they were frequently co-purchased. Vertex features encode the political affiliation of the author (liberal, conservative, or neutral).

We choose to partition the network into $B=3$ communities as we only have this many distinct values for political affiliation (conservative, liberal or neutral).
From Figure~\ref{fig:polbooks-null} we see that all 3 blocks have a distinct political affiliation as their largest positive component.  
Furthermore, in Table~\ref{tab:results} we see that the training and test losses 
are very similar and both are low in magnitude. This is strong evidence 
that political affiliation is a very appropriate explanatory 
variable for the overall network structure.

However, from Figure~\ref{fig:polbooks-accuracy} we see that block 1 has low accuracy. 
This suggests that detected block 1 is not solely composed of ``neutral" books but also 
contains some ``liberal" and ``conservative" authors. Examining 
Figure~\ref{fig:polbooks-graph}, we see the majority of paths between blocks 2 and 3 go through block 1.
Block 1 is in effect a bridge between the ``conservative'' and ``liberal'' blocks so it is unsurprising that some books from either side leak into block 1.

\subsubsection{Primary school dynamic contacts}

\cite{schools} ($N=238, E=5539, D=13$) -- network of face-to-face contacts amongst students and teachers at a primary school in Lyon, France. Vertex features include class membership (one of 10 values: 1A-5B), gender (male, female) and teacher status encoded as an 11th school-class. We choose to analyse just the second day of results.

We choose the number of communities $B=10$, in line with the total number of 
school classes. From Figure~\ref{fig:school-null}, we see only the pupils' class memberships (1A-5B) have survived
the dimensionality-reduction process for $D'=10$;
gender and teacher/student status have been discarded,
meaning these are poor predictors of overall macro-structure.

The vast majority of blocks are composed of a single class. 
However, some blocks have two comparably strong classes as their predictors. 
For example, blocks 2 and 5 both contain classes 3A and 3B as their 2 best predictors. 
This suggests that the social divide between classes is less pronounced 
for pupils in year 3. Conversely, some classes are found to extend over two 
detected blocks (class 2B spans blocks 8 and 9) but we do 
not have a feature which explains the division.

Figure~\ref{fig:school-accuracy} shows excellent accuracy for the majority of blocks. In fact the only blocks with low accuracy are those that have a school-class span two blocks such that we cannot reliably distinguish between the two. This is more pronounced when we apply hard classification rather than the soft cross-entropy loss. Indeed block 1 has low accuracy because class 1A has a higher weight component for block 3 than it does for block 1. As we use hard classification to compute accuracy, vertices belonging to class 1A are predicted to belong to block 3. A similar effect is seen in block 5 which often loses out to block 2. The same can be said of blocks 8 and 9. However, in this case the weights for class 2B are very similar -- thus explaining the roughly evenly split accuracy for blocks 8 and 9.

\subsubsection{Facebook egonet}

\cite{fb-snap} ($N=747, E=30025, D=480$) -- an assortment of Facebook users' friends lists. Vertex features are fully anonymised and encode information about each user's education history, languages spoken, gender, home-town, birthday etc. We focus on the egonet with id 1912.

We choose $B=10$ and $D'=10$ for this experiment. The selected features 
(Figure~\ref{fig:fb-null}) are those that best explain the high-level 
community structure. The majority of them are education related. 
Nevertheless, for $D'=10$ we only have good explanations for some of the detected blocks; several blocks in 
Figure~\ref{fig:fb-null} do not have high-magnitude components for $D'=10$. This is further emphasised by the disparate accuracies in Figure~\ref{fig:fb-accuracy}. Nevertheless, observe that the accuracy is only high for blocks that contain high magnitude weights. The only exception to this rule is block 9 which nonetheless may have high magnitude weights just below the cut-off $c^*$.

When the feature dimension is very large, it becomes increasingly likely that 
a particular feature may uniquely identify a small set of nodes. If these nodes 
are all part of the same community, then the classifier may overfit for that 
particular parameter. The regularisation term imposed by the prior goes some 
way towards alleviating this problem. Nevertheless, we see in 
Figure~\ref{fig:fb-null} that the feature \verb*|birthday-5| has a very 
high weight as it relates to block 1 -- but it is highly unlikely
that birthdays determine graphical structure.
\begin{table}[!h]
	\centering
	\caption{Experimental results averaged over $n=10$ iterations (mean $\pm$ std. dev.).}
	\label{tab:results}
	\resizebox{\textwidth}{!}{%
		\begin{tabular}{c|ccc|c|cc|ccc}
			Dataset  & $B$ & $D$ & $D'$ & $\bar{S}_e$ & $\bar{\Lcal}_0$ & $\bar{\Lcal}_1$ & $c^*$ & $\bar{\Lcal}_0'$ & $\bar{\Lcal}_1'$  \\ \hline
			Polbooks & 3 & 3 & -- & $2.250 \pm 0.000$ & $0.563 \pm 0.042$ & $0.595 \pm 0.089$ & -- & -- & -- \\
			School & 10 & 13 & 10 & $1.894 \pm 0.004$ & $0.787 \pm 0.127$ & $0.885 \pm 0.129$ & $1.198 \pm 0.249$ & $0.793 \pm 0.132$ & $0.853 \pm 0.132$ \\
			FB egonet & 10  & 480 & 10 & $1.626 \pm 0.003$ & $1.326 \pm 0.043$ & $1.538 \pm 0.069$ & $0.94 \pm 0.019$ & $1.580 \pm 0.150$ & $1.605 \pm 0.106$
		\end{tabular}
	}
\end{table}

\begin{figure}[!h]
	\centering
	\begin{subfigure}[t]{0.28\linewidth}
		\centering
		\includegraphics[width=\linewidth]{polbooks-graph.png}
		\caption{Polbooks}
		\label{fig:polbooks-graph}
	\end{subfigure}
	\hfill
	\begin{subfigure}[t]{0.28\linewidth}
		\centering
		\includegraphics[width=\linewidth]{school-graph.png}
		\caption{School}
		\label{fig:school-graph}
	\end{subfigure}
	\hfill
	\begin{subfigure}[t]{0.28\linewidth}
		\centering
		\includegraphics[width=\linewidth]{fb-graph.png}
		\caption{Facebook egonet}
		\label{fig:fb-graph}
	\end{subfigure}
	\begin{subfigure}[t]{0.11\linewidth}
		\centering
		\includegraphics[width=0.8\linewidth]{10-vertical-legend.png}
		\caption{Block Legend}
		\label{fig:10-legend}
	\end{subfigure}
	\caption{Networks laid out and coloured according to inferred block memberships $\hat{y}$ for a given experiment iteration. Visualisation performed using \textit{graph-tool} \cite{peixoto_graph-tool_2014}.}
	\label{fig:graphs-all}
\end{figure}


\begin{figure}[!h]
	\centering
	\begin{minipage}{0.42\linewidth}
		\centering
		\begin{subfigure}{\linewidth}
			\centering
			\imagebox{0.8\linewidth}{\includegraphics[width=\linewidth]{polbooks-null-1}}
			\caption{Political Books.}
			\label{fig:polbooks-null}
		\end{subfigure}
		
		\begin{subfigure}{\linewidth}
			\centering
			\imagebox{0.85\linewidth}{\includegraphics[width=\linewidth]{school-null-1}}
			\caption{Primary School.}
			\label{fig:school-null}
		\end{subfigure}
		
		\begin{subfigure}{\linewidth}
			\centering
			\imagebox{0.9\linewidth}{\includegraphics[width=\linewidth]{fb-null-1}}
			\caption{Facebook Egonet.}
			\label{fig:fb-null}
		\end{subfigure}
		\caption{$\theta$-samples. Dotted line $\pm c^*$.}
	\end{minipage}
	%
	\begin{minipage}{0.42\linewidth}
		\centering
		\begin{subfigure}[t]{\linewidth}
			\centering
			\imagebox{0.8\linewidth}{\includegraphics[width=\linewidth]{polbooks-accuracy-2}}
			\caption{Political Books.}
			\label{fig:polbooks-accuracy}
		\end{subfigure}
		
		\begin{subfigure}{\linewidth}
			\centering
			\imagebox{0.85\linewidth}{\includegraphics[width=\linewidth]{school-accuracy-1}}
			\caption{Primary School.}
			\label{fig:school-accuracy}
		\end{subfigure}
		
		\begin{subfigure}{\linewidth}
			\centering
			\imagebox{0.9\linewidth}{\includegraphics[width=\linewidth]{fb-accuracy-1}}
			\caption{Facebook Egonet.}
			\label{fig:fb-accuracy}
		\end{subfigure}
		\caption{Per-block accuracy $\eta(j)$.}
	\end{minipage}
\end{figure}