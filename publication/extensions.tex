\section{Extensions}

The greatest limitation of the current FFBM formulation is that it can only explain structure at the macro-scale. In other words, we cannot explain structure within each detected block. Future work will benefit from extending the FFBM to be hierarchical in nature. This would be a relatively natural extension. Indeed, the SBM has already been extended to a hierarchical form, often called the nested SBM \cite{SBM-hierarchical}. The idea is to divide each block into sub-blocks and so on recursively until a specified depth is reached. The full block membership for a particular vertex now encodes the memberships at all levels of the hierarchy.

The necessary modification of the feature-to-block generator is also rather natural. Given the nested SBM, we would have a hierarchy of generators, each generating a block membership at a particular level of the hierarchy. Nevertheless, this does pose some practical issues for scalability; supposing we have $L$ levels in our hierarchy and each divides the parent block into $B$ sub-blocks, then the number of distinct generators necessary scales as $O(B^L)$. To avoid exponential growth in the number of model parameters, we could apply some form of dimensionality reduction as we descend the layers so that each generator is only given relevant features as input.