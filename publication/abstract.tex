\begin{abstract}
Labelled networks are a very common and important class of data.
A typical inference goal is to determine how the vertex labels
(or {\em features}) affect the network's structure.

A standard approach has been to partition the network into blocks grouped
by distinct values of the feature of interest. A block-based random
graph model -- typically a variant of the stochastic block model --
is then used to test for evidence of asymmetric behaviour between these
feature-based communities. Nevertheless, the resulting communities
often do not produce a natural partition of the graph.

In this work, we introduce a new generative model, the feature-first block model (FFBM),
which facilitates the use of richer queries on labelled network structure.
We develop a Bayesian framework for inference with this model,
and we present a method to efficiently sample from the posterior
distribution of the FFBM parameters.

We apply the proposed methods to a variety of network data
to extract the most important features along which the vertices
are partitioned. The main advantages of the proposed approach are
that the whole feature-space is used automatically and that features
can be rank-ordered implicitly according to impact.

\keywords{Stochastic Block Models \and Bayesian Inference.}

\end{abstract}