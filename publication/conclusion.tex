\section{Conclusion}
\label{sec:conclusion}

The Feature-First Block Model (FFBM) introduced 
in this paper is a new generative model for labelled networks well-suited for describing how features
affect structure. The idea is to divide the graph into its most natural partition and determine whether 
the vertex features can accurately explain this partition. 
It is easy to find vertex features that are in some way 
correlated with the graphical structure; only when 
we find the feature that best describes the most pronounced partition,
do we have a stronger case for causation.

Using this new model,
we go on to describe an efficient inference algorithm to sample 
the parameters of the FFBM. 
This is shown to be effective at extracting and describing 
the most natural communities in a labelled network. Nevertheless, the approach 
can only currently explain the structure at the macro-scale. Future work will benefit from extending 
the FFBM to be hierarchical in nature. That way, the structure of the network 
can be explained at all length-scales of interest.
