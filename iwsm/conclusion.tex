\section{Conclusion}
\label{sec:conclusion}

The feature-first block model
(FFBM) is introduced, as a 
new generative model for labelled 
networks with communities.
An efficient MCMC algorithm  
is developed for sampling 
from the posterior distribution of
the relevant parameters in the FFBM;
the main idea is to divide up the graph into 
its most natural partition under the associated
parameter values, and then to determine whether 
the vertex features can accurately explain the partition. 
Through applications on empirical
network data, this approach 
is demonstrated to be effective at extracting and describing 
the most natural communities in a labelled network. 
Nevertheless, it
can only currently explain the structure at the macroscopic
scale. Future work will benefit from extending 
the FFBM to a further hierarchical model,
so that
the structure of the network 
can be explained at all scales of interest.


