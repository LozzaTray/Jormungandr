\section{Experimental results}
\label{sec:experiments}

We apply our proposed methods to a variety of labelled networks:

\begin{itemize}
	\item \textbf{Political books} \cite{polbooks} ($N=105, E=441, D=3$) -- network of Amazon political books published close to the 2004 presidential election. Two books are connected if they were frequently co-purchased. Vertex features encode the political affiliation of the author (liberal, conservative, or neutral).
	\item \textbf{Primary school dynamic contacts} \cite{schools} ($N=238, E=5539, D=13$) -- network of 238 individuals (students and teachers),
with edges denoting face-to-face contacts at a primary school in Lyon, France. 
The vertex features are class membership (one of 10 values: 1A-5B), gender (male, female), and status (teacher, student). We choose to analyse just the second day of results.
	\item \textbf{Facebook egonet} \cite{fb-snap} ($N=747, E=30025, D=480$) -- network of Facebook users with edges denoting ``friends''.
Vertex features are fully anonymised and encode information about each user's education history, languages spoken, gender, home-town, birthday etc. We focus on the egonet with id 1912.
\end{itemize}
%
For reference, the inferred partitions for all of these are given on Figure~\ref{fig:graphs-all}.
We employ the following metrics to assess model performance. 
First, the average
description length per entity (nodes and edges) 
$\bar{S}_e$ 
used to gauge the SBM fit is defined as:
%
\begin{equation}
	\bar{S}_e \triangleq \frac{1}{(N+E) |\mathcal{T}_b|} \sum_{t\in \mathcal{T}_b} S \left( b^{(t)} \right).
	\label{eqn:mean-dl}
\end{equation}
%
Next, to assess the performance of the feature-to-block predictor, 
the vertex set $[N]$ 
is partitioned at random so that 
a constant fraction $f$ of vertices form the training set $\mathcal{G}_0$ and 
the remainder form the test set $\mathcal{G}_1$.
The $b$-chain is run using the whole network but only vertices $v \in \mathcal{G}_0$ are 
used for the $\theta$-chain. 
Then the  the average cross-entropy loss 
over each set is used to gauge the quality of the fit,
%
\begin{equation}
	\bar{\mathcal{L}}_\star \triangleq \frac{1}{|\mathcal{T}_\theta|} \sum_{t \in \mathcal{T}_\theta} \mathcal{L}_\star^{(t)},
	\quad \textrm{where} \quad
	\mathcal{L}_\star^{(t)} \triangleq \frac{1}{|\mathcal{G}_\star|} \sum_{i \in \mathcal{G}_\star}\sum_{j \in [B]} \hat{y}_{ij} \log \frac{1}{\phi_j \left(x_i; \theta^{(t)} \right)},
	\label{eqn:cross-entropy-loss}
\end{equation}
%
where $\star \in \{0, 1\}$ toggles between the training and test sets
and $\hat{y}_{ij}$ is defined in~(\ref{eqn:y-hat}).
%
Nevertheless, the cross-entropy loss is a coarse measure of fit. 
A new measure, specific to each detected block,
can be defined as follows. Let
$\mathcal{B}_\star(j)$ 
be the set of vertices with maximum a posteriori probability of belonging 
to block~$j$,
$
	\mathcal{B}_\star(j) \triangleq \{i \in \mathcal{G}_\star : \hat{b}_i = j\},
$
where
$ 
	\hat{b}_i \triangleq \underset{j}{\operatorname{argmax}}\hat{y}_{ij},
$
and
define the {\em block-accuracy} for block $j$ as,
%
\begin{equation}
	\eta_\star(j) \triangleq \frac{1}{|\mathcal{B}_\star (j)| \cdot 
	|\mathcal{T}_\theta| } 
	\sum_{i \in \mathcal{B}_\star (j)}  \sum_{t \in \mathcal{T}_\theta}
	\boldsymbol{1} \left\{\hat{b}_i = \underset{j}{\operatorname{argmax}}\phi_j \left( x_i; \theta^{(t)} \right) \right\}.
	\label{eqn:accuracy}
\end{equation}
%
This effectively tests whether the feature-to-block and 
graph-to-block predictions agree in their largest component.
For the higher-dimensional datasets, we also apply the 
dimensionality reduction method 
of Section~\ref{sec:dim-reduction}.  
We then retrain the feature-block predictor using only the retained 
feature set $\mathcal{D}'$, and report the log-loss over the training and 
test sets for the reduced classifier -- 
denoted $\bar{\mathcal{L}}_0'$ and $\bar{\mathcal{L}}_1'$ respectively. 
