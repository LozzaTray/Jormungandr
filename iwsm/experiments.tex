\section{Experimental results}
\label{sec:experiments}

We apply our proposed methods to a variety of real-world datasets.
For reference, the inferred partitions $b$ for all of these are given on Figure~\ref{fig:graphs-all}.
We require metrics to assess model performance. 
First, the average
description length per entity (nodes and edges) 
$\bar{S}_e$ 
is used to gauge the SBM fit.
%%
%\begin{equation}
%	\bar{S}_e \triangleq \frac{1}{(N+E) |\mathcal{T}_b|} \sum_{t\in \mathcal{T}_b} S \left( b^{(t)} \right).
%	\label{eqn:mean-dl}
%\end{equation}
%%
Second, to assess the performance of the feature-to-block predictor, 
the vertex set $[N]$ 
is partitioned at random into training and test sets, $\mathcal{G}_0$ and $\mathcal{G}_1$. 
Then the average cross-entropy loss 
over each set is used to gauge the quality of the fit, $\bar{\mathcal{L}}_\star$,
%%
%\begin{equation}
%	\bar{\mathcal{L}}_\star \triangleq \frac{1}{|\mathcal{T}_\theta|} \sum_{t \in \mathcal{T}_\theta} \mathcal{L}_\star^{(t)},
%	\quad \textrm{where} \quad
%	\mathcal{L}_\star^{(t)} \triangleq \frac{1}{|\mathcal{G}_\star|} \sum_{i \in \mathcal{G}_\star}\sum_{j \in [B]} \hat{y}_{ij} \log \frac{1}{\phi_j \left(x_i; \theta^{(t)} \right)},
%	\label{eqn:cross-entropy-loss}
%\end{equation}
%%
%where $\star \in \{0, 1\}$ toggles between the training and test sets.
%
%and $\hat{y}_{ij}$ is defined in~(\ref{eqn:y-hat}).
%
%Nevertheless, the cross-entropy loss is a coarse measure of fit. 
%A new measure, specific to each detected block,
%can be defined as follows. Let
%$\mathcal{B}_\star(j)$ 
%be the set of vertices with maximum a posteriori probability of belonging 
%to block~$j$,
%$
%	\mathcal{B}_\star(j) \triangleq \{i \in \mathcal{G}_\star : \hat{b}_i = j\},
%$
%where
%$ 
%	\hat{b}_i \triangleq \underset{j}{\operatorname{argmax}}\hat{y}_{ij},
%$
%and
%define the {\em block-accuracy} for block $j$ as,
%%
%\begin{equation}
%	\eta_\star(j) \triangleq \frac{1}{|\mathcal{B}_\star (j)| \cdot 
%	|\mathcal{T}_\theta| } 
%	\sum_{i \in \mathcal{B}_\star (j)}  \sum_{t \in \mathcal{T}_\theta}
%	\boldsymbol{1} \left\{\hat{b}_i = \underset{j}{\operatorname{argmax}}\phi_j \left( x_i; \theta^{(t)} \right) \right\}.
%	\label{eqn:accuracy}
%\end{equation}
%%
%This effectively tests whether the feature-to-block and 
%graph-to-block predictions agree in their largest component.
%
For the higher-dimensional datasets, we also develop and apply a novel 
dimensionality reduction method to select only the top $D'$ features.  
We then retrain the feature-block predictor using only the retained 
feature set, and report the cross-entropy loss over the training and 
test sets for the reduced classifier -- 
denoted $\bar{\mathcal{L}}_\star'$.

Table~\ref{tab:results} summarises the high-level results for each experiment. 
We see that the dimensionality reduction procedure 
brings the training and test losses closer. This indicates that
the retained features
are indeed well correlated with the underlying graphical 
partition and that the approach generalises correctly. 
%The test loss variance is higher than the training loss variance as the test set is smaller and so more susceptible to variability in its construction.
%%
%The average description length per entity of the graph,
%$\bar{S}_e$,
%has very small variance, suggesting that
%the detected communities can be found reliably (to within an arbitrary 
%relabelling of blocks). 
