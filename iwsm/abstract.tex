\abstract{
Labelled networks are an important class of data,
naturally appearing
in numerous applications in science and engineering.
A typical inference goal is to determine how the vertex labels
(or {\em features}) affect the network's structure.
In this work, we introduce a new generative model, the feature-first block model (FFBM),
that facilitates the use of rich queries on labelled networks.
We develop a Bayesian framework and devise a two-level Markov chain Monte 
Carlo approach to efficiently sample from the
relevant posterior distribution of the FFBM parameters. This allows us to infer if and how the observed vertex-features affect macro-structure.
We apply the proposed methods to several real-world networks
to extract the most important features along which the vertices
are partitioned. The approach stands out from its peers in
that the whole feature-space is used automatically and features
can be rank-ordered implicitly by importance.
}
