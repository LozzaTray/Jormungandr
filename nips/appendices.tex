\section{Appendix}

\subsection{Derivation of conditional block distribution given feature matrix}
\label{appdx:b|x}

We wish to determine the form of $p(b| X)$. This can be done by integrating over the joint probability with respect to $\theta$.
%
\begin{align*}
	p(b | X) &= \int p(b , \theta| X, \theta) d\theta = \int p(b | X, \theta) p(\theta | X) d\theta \\
	&=\int p(b | X, \theta) p(\theta) d\theta = \int \prod_{i=1}^{N} \phi_{b_i}(x_i; \theta) p(\theta) d\theta \\
	&= \prod_{i=1}^{N} \int \frac{\exp(w_{b_i}^T \tilde{x}_i) \prod_{j=1}^{B} \Gaussian(w_j; 0, \sigma_\theta^2 I)}{\sum_{k=1}^{B} \exp(w_{k}^T \tilde{x}_i)} dw_{1:B}
\end{align*}
%
We note that $b_i \in {1, 2, \dots B}$ and so the integral's value is unchanged with respect to $b_i$. The integrand has the same form no matter which value $b_i$ takes as the prior is the same for each $w_j$. As such the integral can only be a function of at most $\tilde{x}_i$ and $\sigma_\theta^2$ as it is symmetric with respect to $b_i$ and all the various $w_j$ are integrated out as they are dummy variables. Therefore, denoting the integral by the (unknown) function $f(\tilde{x}_i, \sigma_\theta^2)$, we write $p(b| X)$ as follows:
%
\begin{align*}
	p(b | X) &= \prod_{i=1}^{N} f(\tilde{x}_i, \sigma_\theta^2) = \textrm{const w.r.t } b = c
\end{align*}
%
As this is a constant with respect to $b$ we conclude that $p(b | X)$ must be a uniform distribution. $\nicefrac{1}{c}$ is simply the size of the set of values that $b$ can take. We know $b_i \in \mathcal{B} = \{1, 2, \dots B\}$. Therefore, $b \in \mathcal{B}^N$ and $|\mathcal{B}^N| = |\mathcal{B}|^N = B^N= \nicefrac{1}{c}$. Putting this all together we show that:
%
\begin{equation}
	p(b | X) = B^{-N}
\end{equation}

\subsection{Derivation of U gradient with respect to feature parameters}
\label{appdx:gradu}
The goal is to determine $\nabla U(\theta)$, the gradient of the negative log posterior with respect to the parameters. We repeat the form of $U(\theta)$ in equation \ref{eqn:U-form-appdx}.
%
\begin{equation}
	U(\theta) = \left( \sum_{i=1}^{N} \sum_{j=1}^{B} y_{ij} \log \frac{1}{a_{ij}} \right)
	+ \frac{1}{2\sigma_\theta^2} ||\theta||^2
	\label{eqn:U-form-appdx}
\end{equation}
%
Where $y_{ij}$ is independent of $\theta$ and $a_{ij}$ is the output from the softmax layer, with form as given in equation \ref{eqn:a-ij}.
%
\begin{equation}
	a_{ij} \coloneqq \phi_{j} (x_i; \theta) = \frac{\exp(w_j^T \tilde{x}_i)}{\sum_{b=1}^{B} \exp(w_b^T \tilde{x}_i)}
	\label{eqn:a-ij} 
\end{equation}
%
We note that $\theta = \{w_k\}_{k=1}^B$, and as such we can write this in vector form $\theta = \left[w_1^T, w_2^T \dots w_B^T  \right]^T$. Therefore, $\nabla U(\theta) = \left[\nicefrac{\partial U}{\partial w_1}^T,\nicefrac{\partial U}{\partial w_2}^T \dots \nicefrac{\partial U}{\partial w_B}^T  \right]^T$; to compute $\nabla U(\theta)$ it suffices to find the form of $\nicefrac{\partial U}{\partial w_k}$ with respect to a general $k$.

To this end, we must first find partial derivatives of $a_{ij}$ and $||\theta||$ with respect to $w_k$. Starting with $a_{ij}$:
%
\begin{align}
	\frac{\partial a_{ij}}{\partial w_k} &= \frac
	{\tilde{x}_i \exp(w_j^T \tilde{x}_i) \delta_{jk} \cdot \sum_{b=1}^{B} \exp(w_b^T \tilde{x}_i) 
		- 
		\exp(w_j^T \tilde{x}_i) \cdot \tilde{x}_i \exp(w_k^T \tilde{x}_i)}
	{\left( \sum_{b=1}^{B} \exp(w_b^T \tilde{x}_i) \right)^2} \nonumber \\
	&= \tilde{x}_i \left( a_{ij} \delta_{jk} - a_{ij}a_{ik} \right) 
\end{align}
%
Where $\delta_{jk} \coloneqq \one \left\{ j = k \right\}$. Now moving onto the derivative of $||\theta||^2$:
%
\begin{equation}
	\frac{ \partial}{\partial w_k} ||\theta||^2 = \frac{\partial}{\partial w_k} \left( \sum_{b=1}^B ||w_b||^2 \right) = 2w_k
\end{equation}
%
We are ready to put this all together, to find the partial derivative of $U(\theta)$ with respect to each $w_k$:
\begin{align}
	\frac{\partial U}{\partial w_k} &= 
	\sum_{i=1}^{N} \sum_{j=1}^{B} y_{ij} 
	\left( \frac{-\tilde{x}_i}{a_{ij}} \left( a_{ij} \delta_{jk} - a_{ij} a_{ik} \right) \right)
	+ \frac{w_k}{\sigma_\theta^2} \nonumber \\
	&=  - \left( \sum_{i=1}^{N} \tilde{x}_i \left( y_{ik} - a_{ik} \sum_{j=1}^{B} y_{ij} \right)
	- \frac{w_k}{\sigma_\theta^2} \right) \nonumber \\
	&= - \left( \sum_{i=1}^{N} \Big\{ \tilde{x}_i (y_{ik} - a_{ik}) \Big\} - \frac{w_k}{\sigma_\theta^2} \right)
\end{align}
%
This is the required result. This form can be computed efficiently through matrix operations. The only property of $y_{ij}$ we have used in the derivation is the sum-to-one constraint $\sum_{j=1}^{B} y_{ij} = 1$ for all $i$.

\subsection{Choosing the MALA step-size}
\label{appdx:step-size}

For sampling from the $\theta$-chain of the block membership generator parameters, we employed the Metropolis Adjusted Langevin Algorithm (MALA). At iteration $t$, the proposed sample is generated by:
%
\begin{equation}
	\theta' = \theta^{(t)} - h_t \nabla U(\theta^{(t)}) + \sqrt{2h_t} \cdot \xi
\end{equation}
%
There are two competing objectives when choosing the step-size $h_t$. On the one hand, we want the step-size to be large so that we arrive at a high density region quickly. However, too large a step-size will lead to a lower acceptance ratio and thus inefficient sampling. A solution to this problem would be to slowly decrease the step-size with $t$ - often called simulated annealing. Therefore, we still have a short burn-in time but will not bounce around the mode for large $t$. As well as the trivial constraint for $h_t$ to be strictly positive and we introduce two further constraints as outlined by \citet{Bayesian-SGLD}:
%
\begin{equation}
	\sum_{t=1}^{\infty} h_t = \infty \qquad \textrm{and} \qquad
	\sum_{t=1}^{\infty} h_t^2 < \infty
	\label{eqn:h-constraints}
\end{equation}
%
The first constraint ensures that we have cover sufficient distance to arrive at any arbitrary point in our domain, no matter the starting point. The second constraint ensures that once we converge to the mode rather than simply bouncing around it. \citet{Bayesian-SGLD} propose the following form for a polynomially decaying step-size which we adopt:
%
\begin{equation}
	h_t = a(b + t)^{-\gamma}
\end{equation}
%
Where $a, b, \gamma$ are hyper-parameters to be chosen. We require $a,b > 0$ and $\gamma \in (0.5, 1]$ to satisfy equation \ref{eqn:h-constraints}. We find empirically that $a=250, b=1000, \gamma=0.8$ yield good burn-in times for our MALA implementations.