\section{Appendix}

\subsection{SBM prior choice explanation}
\label{appdx:prior}

Here we recall for reference
the priors $p(\psi | b)$ from \cite{Peixoto-Bayesian-Microcanonical}:
%
\begin{equation}
	p(\psi_e=e, \psi_k=k | b) = p(e | b) p(\psi_k | e, b) = \left[ \specialchoose{ \specialchoose{B}{2} }{ E} \right]^{-1} 
	\cdot \left[ \prod_r \frac{\prod_j \eta_j^r !}{n_r! q(e_r, n_r)} \right],
\end{equation}
%
where $\specialchoose{n}{m}$ is shorthand 
for $\binom{n+m-1}{m} = \frac{(n+m-1)!}{(n-1)!(m)!}$,
which can be thought of as the total number of distinct histograms 
produced by $m$ samples in $n$ bins.
The value
$E = \frac{1}{2} \sum_{r,s} e_{rs}$ is the total number of edges in the graph. 
Importantly, $E$ is not allowed to vary and so $p(e|b)$ is uniform in $e$.
The variable $\eta_j^r$ denotes the number of vertices in block $r$ 
that have degree $j$; formally, $\eta_j^r \coloneqq \sum_{i} \one\left\{b_i = r \right\} \one \left\{k_i = j \right\}$. 
The denominator $q(m, n)$ denotes the number of different histograms 
produced by $m$ samples in 
at most $n$ non-zero bins that sum to $m$. 
Finally, $e_r \coloneqq \sum_{s} e_{rs}$ is the total number 
of half edges in block $r$ and $n_r \coloneqq \sum_{i} \one\{b_i = r\}$ 
is the number of vertices assigned to block $r$. 

These were chosen carefully in \cite{Peixoto-Bayesian-Microcanonical} to 
more closely match the structure of empirical networks than simple 
uniform priors. We do not repeat these arguments here.

\subsection{Derivation of conditional block distribution given feature matrix}
\label{appdx:b|x}

We determine the form of $p(b| X)$ by integrating
out the parameters $\theta$. From the definitions we have:
%
\begin{align*}
	p(b | X) &= \int p(b , \theta| X, \theta) d\theta = \int p(b | X, \theta) p(\theta | X) d\theta \\
	&=\int p(b | X, \theta) p(\theta) d\theta = \int \prod_{i \in [N] } \phi_{b_i}(x_i; \theta) p(\theta) d\theta \\
	&= \prod_{i \in [N]} \int \frac{\exp(w_{b_i}^T \tilde{x}_i) \prod_{j \in [B]} \Gaussian(w_j; 0, \sigma_\theta^2 I)}{\sum_{k \in [B]} \exp(w_{k}^T \tilde{x}_i)} dw_{1:B}.
\end{align*}
%
The key observation here is that 
the value of the integral is independent
of the value of $b_i \in [B]$:
The integrand has the same form regardless of $b_i$,
because the prior is the same for each $w_j$. 
Therefore, the integral is only a function of $\tilde{x}_i$ and $\sigma_\theta^2$,
which means that, as function of $b$, $p(b|X)\propto 1$. And as
$b$ takes values in $[B]^N$, we necessarily have:
%
\begin{equation}
	p(b | X) = \frac{1}{\big|[B]^N\big|}=B^{-N}.
\end{equation}

\subsection{Derivation of {\boldmath $\;U(\theta)$} }
\label{appdx:form-U}

Recall from~(\ref{eq:U}) is Section~\ref{s:sfb} that,
$$	\pi_\theta(\theta) \propto p(\theta | X, b) \propto p(b | X, \theta) p(\theta) \propto  \exp \left( - U(\theta) \right),
$$ 
so that $U$ can be expressed as,
$$
	U(\theta) 
= - \left( \log p(b | X, \theta) + \log p(\theta) \right) + \textrm{const}.
$$
Writing,
$y_{ij} \coloneqq \one \left\{ b_i = j \right\}$ and 
$a_{ij} \coloneqq \phi_j(x_i; \theta)$, we have,
%
\begin{equation}
	\log p(b | X, \theta) = \sum_{i \in [N]} \sum_{j \in [B]} y_{ij} \log a_{ij}  \quad \textrm{and} \quad
	\log p(\theta) = -\frac{(D+1)(B)}{2} \log 2\pi - \frac{1}{2 \sigma_\theta^2} 
\|\theta \|^2,
	\label{eqn:U-constituent-terms}
\end{equation}
%
where
$\|\theta\|^2 = \sum_{i} \theta_{i}^2 = \sum_{j=1}^{B} \|w_j\|^2$ 
is the Euclidean norm of the vector of parameters $\theta$.
Therefore, discarding constant terms, we 
obtain exactly the representation~(\ref{eqn:U-form}), as claimed.

\subsection{Derivation of {\boldmath $\;\nabla U(\theta)$}}
\label{appdx:gradu}
Here we show how the gradient 
$\nabla U(\theta)$ can be computed explicitly.
Recall the expression for $U(\theta)$ in~(\ref{eqn:U-form}).
Writing $\theta$ as
$\theta = \left[w_1^T, w_2^T \dots w_B^T  \right]^T$,
in order to compute the gradient
$\nabla U(\theta)$ we need to compute
each of its components,
$\nicefrac{\partial U}{\partial w_k}$,
$1\leq k\leq B$.
To that end, we first compute,
%
\begin{align}
	\frac{\partial a_{ij}}{\partial w_k} &= \frac
	{\tilde{x}_i \exp(w_j^T \tilde{x}_i) \delta_{jk} \cdot \sum_{r \in [B]} \exp(w_r^T \tilde{x}_i) 
		- 
		\exp(w_j^T \tilde{x}_i) \cdot \tilde{x}_i \exp(w_k^T \tilde{x}_i)}
	{\left( \sum_{r \in [B]} \exp(w_r^T \tilde{x}_i) \right)^2} \nonumber \\
	&= \tilde{x}_i \left( a_{ij} \delta_{jk} - a_{ij}a_{ik} \right), 
	\label{eq:dadw}
\end{align}
%
where $\delta_{jk} \coloneqq \one \left\{ j = k \right\}$,
and we also easily find,
%
\begin{equation}
	\frac{ \partial}{\partial w_k} \|\theta\|^2 = \frac{\partial}{\partial w_k} \left( \sum_{r \in [B]} \|w_r\|^2 \right) = 2w_k.
\label{eq:dtsdw}
\end{equation}
%
Using~(\ref{eq:dadw}) and~(\ref{eq:dtsdw}), we obtain,
\begin{align}
	\frac{\partial U}{\partial w_k} &= 
	\sum_{i=1}^{N} \sum_{j=1}^{B} y_{ij} 
	\left( -\frac{\tilde{x}_i}{a_{ij}} \left( a_{ij} \delta_{jk} - a_{ij} a_{ik} \right) \right)
	+ \frac{w_k}{\sigma_\theta^2} \nonumber \\
	&=  - \left( \sum_{i=1}^{N} \tilde{x}_i \left( y_{ik} - a_{ik} \sum_{j=1}^{B} y_{ij} \right)
	- \frac{w_k}{\sigma_\theta^2} \right) \nonumber \\
	&= - \left( \sum_{i=1}^{N} \Big\{ \tilde{x}_i (y_{ik} - a_{ik}) \Big\} - \frac{w_k}{\sigma_\theta^2} \right).
\end{align}
%
This can be computed 
efficiently through matrix operations. The only property of $y_{ij}$ 
we have used in the derivation is the constraint $\sum_{j=1}^{B} y_{ij} = 1$,
for all $i$.

\subsection{Choosing the MALA step-size}
\label{appdx:step-size}

Recall that in 
Section~\ref{s:sfb} we used 
the Metropolis Adjusted Langevin Algorithm (MALA)
in order to 
sample from the $\theta$-chain of the block membership 
generator parameters.
At iteration $t$, the proposed sample is generated by:
%
\begin{equation}
	\theta' = \theta^{(t)} - h_t \nabla U(\theta^{(t)}) + \sqrt{2h_t} \cdot \xi.
\end{equation}
%
There are two competing objectives when choosing the step-size $h_t$. 
On the one hand, $h_t$ needs to be large so that the sampler
arrives at a high density region quickly,
while too large a step-size would lead to low acceptance rates and thus 
inefficient sampling. An effective strategy is
to use {\em simulated annealing}: allow $h_t$ to slowly decrease
with $t$, as long as $h_t>0$ for all $t$ and also:
%
\begin{equation}
	\sum_{t=1}^{\infty} h_t = \infty, \qquad \textrm{and} \qquad
	\sum_{t=1}^{\infty} h_t^2 < \infty.
	\label{eqn:h-constraints}
\end{equation}
%
Following \citet{Bayesian-SGLD}, we adopt the 
polynomially decaying step-sizes,
%
$h_t = \alpha(\beta + t)^{-\gamma}$,
%
where $\alpha>0$, $\beta>0$ and $\gamma\in(1/2,1]$ are hyper-parameters.
We make the specific choices,
%
\begin{equation}
	\alpha = \frac{250 \cdot s}{N}, \qquad \beta = 1000, \qquad \gamma = 0.8,
	\label{eqn:step-size-params}
\end{equation}
%
where $N$ is the number of data-points and $s$,
the {\em step-size scaling}, is the only free parameter.

% For approximate methods, we can choose to bypass the MH accept-reject entirely to speed up computation. If this is done, the algorithm is instead called stochastic gradient Langevin diffusion (SGLD) \cite{Bayesian-SGLD}. This speeds up computation at the expense of exactness of the method.

\subsection{Burn-in and thinning}
\label{appdx:burn-in-thinning}

% As with any MCMC method, we must deal with the 
% issues presented by burn-in and thinning. 

When sampling from the $b$- and $\theta$-chains described
in Section~\ref{sec:inference}, we generate
$T_b$ and $T_\theta$ samples total, respectively.
We discard an initial proportion $\kappa_\star\in(0,1)$ of the samples 
as corresponding to a ``burn-in'' period required for the distribution 
of the chain to reach a distribution close to our target,
and we also ``thin'' the remaining samples to 
obain a less-dependent version. For $\star\in\{b,\theta\}$,
the remaning sample sets are denoted $\Tcal_\star$
in the notation of Section~\ref{s:ss},
%
\begin{equation}
	\Tcal_\star = \{T_\star \kappa_\star + i \lambda_\star :  
	0 \leq i \leq \lfloor T_\star(1 - \kappa_\star) / \lambda_\star \rfloor \},
\end{equation}
%
where $\lambda_\star$ controls the thinning. The choice of
$\kappa_\star$ can be determined by plotting the log-target (either $S(b^{(t)})$ 
or $U(\theta^{(t)})$ as a function of $t$,
and choosing $\kappa_\star$ to encompass the region where the log-target has 
roughly reached equilibrium. As we do not leverage sample independence,
$\lambda_\star$ can be chosen less rigorously; we often simply
use $\lambda_b=5$ and $\lambda_\theta = 10$.

\subsection{Initializing the b-chain}

For the purposes of the FFBM model, the number of blocks $B$ is a constant 
which must be specified. If the choice of $B$ is influenced 
by the observed data, then the analysis is no longer ``fully Bayesian''
and belongs to the class of methods referred to as ``empirical Bayes.''
However, as the number of blocks only specifies the coarseness of the 
analysis, it is reasonable to allow it to vary. Indeed, 
\citet{peixoto-determine-B} shows that for a fixed 
average degree the maximum number of detectable blocks scales 
as $O(\sqrt{N})$ where $N$ is the number of vertices.

If $B$ is allowed to vary in the $b$-chain (i.e.,
when new blocks can be created and empty blocks are allowed),
then the chain can be run until a minimum description 
length (MDL) solution is reached. We take the number of non-empty blocks 
in the MDL solution to be our fixed block number $B$ for subsequent analysis. 
Indeed, it is prudent to start the $b$-chain at this MDL solution as then 
the necessary burn-in time can be greatly reduced.

\subsection{Algorithms}
\label{appdx:algorithms}

\begin{algorithm} % enter the algorithm environment
	\caption{Block membership sample generation} % give the algorithm a caption
	\label{alg:b-samples} % and a label for \ref{} commands later in the document
	\begin{algorithmic} % enter the algorithmic environment
		\State $b^{(0)} \gets \argmin_b S(b|A)$ \Comment{Implemented as greedy heuristic in \textit{graph-tool} library}
		\For{$t \in \{0, 1 \dots T_b - 1\}$}
		\State $b' \gets \sim q_b(b^{(t)}, b' | A)$
		\State $\log \alpha_b \gets \log \alpha_b(b^{(t)}, b' | A)$
		\State $\eta \gets \sim \textrm{Unif}(0,1)$
		\If{$\log \eta < \log \alpha_b$}
		\State $b^{(t+1)} \gets b'$
		\Else
		\State $b^{(t+1)} \gets b^{(t)}$
		\EndIf
		\EndFor
		
		\State \textbf{return} $\{b^{(t)}\}_{t=1}^{T_b}$
		
	\end{algorithmic}
\end{algorithm}

\begin{algorithm} % enter the algorithm environment
	\caption{FFBM parameter pseudo-marginal inference} % give the algorithm a caption
	\label{alg:theta-samples} % and a label for \ref{} commands later in the document
	\begin{algorithmic} % enter the algorithmic environment
		
		\State $\hat{Y}_{ij} \gets \frac{1}{|\Tcal_b|} \sum_{t \in \Tcal_b} \one \{ b^{(t)}_i = j\} \quad \forall i,j$
		\State $\theta^{(0)} \gets \sim \Gaussian(0, \sigma_\theta I)$
		
		\item[]
		
		\For{$t \in \{0, 1 \dots T_\theta - 1\}$}
		\State $\xi \gets \sim \Gaussian(0, I)$
		\State $\theta' \gets \theta^{(t)} - h_t \nabla U(\theta^{(t)} | X, \hat{Y}) + \sqrt{2h_t} \cdot \xi$
		\State $\log \alpha_\theta \gets \log \alpha_\theta(\theta^{(t)}, \theta' | A, \hat{Y})$
		\State $\eta \gets \sim \textrm{Unif}(0,1)$
		\If{$\log \eta < \log \alpha_\theta$}
		\State $\theta^{(t+1)} \gets \theta'$
		\Else
		\State $\theta^{(t+1)} \gets \theta^{(t)}$
		\EndIf
		\EndFor
		
		\State \textbf{return} $\{\theta^{(t)}\}_{t=1}^{T_\theta}$
		
	\end{algorithmic}
\end{algorithm}

\FloatBarrier
\subsection{Hyperparameter values}
\label{appdx:hyperparams}

\begin{table}[!h]
	\centering
	\caption{Hyper-parameter values for each experiment.}
	\label{tab:hyperparams}
	\resizebox{\textwidth}{!}{%
		\begin{tabular}{c|ccc|ccc|cccc|cc|cccc}
			Dataset & 
			$B$ & $f$ & $\sigma_\theta$ & 
			$T_b$ & $\kappa_b$ & $\lambda_b$ & 
			$T_\theta$ & $\kappa_\theta$ & $\lambda_\theta$ & $s$ &
			$k$ & $D'$ &
			$T_\theta'$ & $\kappa_\theta'$ & $\lambda_\theta'$ & $s'$
			\\ \hline
			Polbooks &
			3 & 0.7 & 1 &
			1,000 & 0.2 & 5 &
			10,000 & 0.4 & 10 & 0.05 &
			-- & -- & 
			-- & -- & -- & -- \\
			School &
			10 & 0.7 & 1 &
			1,000 & 0.2 & 5 &
			10,000 & 0.4 & 10 & 0.2 &
			1 & 10 & 
			10,000 & 0.4 & 10 & 0.2 \\
			FB Egonet &
			10 & 0.7 & 1 &
			1,000 & 0.2 & 5 &
			10,000 & 0.4 & 10 & 0.017 &
			1 & 10 & 
			10,000 & 0.4 & 10 & 0.5 \\
		\end{tabular}
	}
\end{table}

\subsection{Implementation details}
\label{appdx:imp-details}

All data analysis and visualisation was implemented in Python. Full source code is available in the supplementary material. The scripts were run using a standard PC using the Windows Subsystem for Linux (WSL) environment. Specs are:

\begin{itemize}
	\item \textbf{CPU}: Intel(R) Core(TM) i7-1065G7
	\item \textbf{RAM}: 8GB
	\item \textbf{GPU}: Intel(R) Iris(R) Plus Graphics
\end{itemize}

On this hardware each experiment iteration took the following amount of time to execute:

\begin{table}[!h]
	\centering
	\caption{Compute-time for each experiment.}
	\label{tab:compute-time}
	\begin{tabular}{c|ccc|c}
		Dataset & $b$-chain & $\theta$-chain & Reduced $\theta$-chain & Overall compute time \\ \hline
		Polbooks & $\sim$1s & $\sim$4s & -- & $\sim$5s \\
		School & $\sim$10s & $\sim$10s & $\sim$10s & $\sim$30s \\
		FB Egonet & $\sim$20s & $\sim$180s & $\sim$10s & $\sim$210s
	\end{tabular}
\end{table}

