\section{Introduction}

A somewhat surprising property of many real-world networks is that they exhibit strong community structure. In other words, each node will often belong to a cluster of densely connected nodes. There is high interest in recovering the latent communities from the observed graphs. The inferred communities can be exploited for compression algorithms \cite{cluster-compression} or used for link prediction in incomplete networks \cite{link-prediction} to name but a few applications.

Nevertheless, we restrict our analysis to labelled networks. These are graphs where we have information about the properties of each node. We shall refer to these node properties as features. One of the most common questions we can ask of labelled networks is whether a given vertex feature has an impact on structure of the graph. To answer this question from a Bayesian perspective we must use a random graph model; the most common form is called the stochastic block model (SBM) \cite{vanilla-sbm}. This is a latent variable model where each vertex belongs to a single block and the probability two nodes are connected depends only on the. There have been many variants to this model; the most popular are the mixed-membership stochastic block model (MMSBM) \cite{mixed-membership-sbm} and overlapping stochastic block model (OSBM) \cite{overlapping-sbm}. Effectively, these just extend the model to allow each vertex to belong to multiple blocks simultaneously.

However, a major drawback of these graphical models as applied to labelled networks is that they do not automatically include vertex features in the random graph generation process. Approaches based on graph neural networks \cite{sbm-gnn} that utilise vertex features have been developed but these lack the easy interpretability of the simpler models. To analyse a labelled network using one of the simple SBM variants, a typical inference procedure would be to partition the graph into blocks grouped by distinct values of the feature of interest. The associated model can then be used to test for evidence of heterogeneous connectivity between the feature-grouped blocks. Nevertheless, this approach is limited in that it can only consider one feature at a time. Furthermore, it is hard to rank order the magnitude of the impact of each feature if they are considered independently. We would instead prefer to partition the graph into its most natural blocks and then determine which feature best predict the resulting partition.

With these desiderata in mind, we present a novel model for modelling labelled networks, which we call the feature-first block model (FFBM). This allows us to use the latent block membership as a stepping stone in our analysis. Therefore, we extract which features have the largest impact on overall graphical structure. This can be thought of as a form of dimensionality-reduction where we only keep the node features that are useful in predicting community memberships.