\section{Introduction}

There is a wealth of networks in the world. Some examples of this graphical data are social networks, website hyperlinks and academic collaboration with more being produced each second. It is clear we need tools to analyse this increasingly omni-present form of data.

A somewhat surprising property of real-world graphs is that they exhibit strong community structure. In other words, each node will often belong to a cluster of densely connected nodes. This property is often exploited by graph compression algorithms and there is high interest in recovering the communities from the observed graph.

A common subset of graphical data is the labelled network. This is a graph where we have information about the properties of each node. We shall refer to these node properties as features. One of the most common questions we can ask of this dataset is what features have the largest impact on the structure of the graph. For example, when analysing an academic collaboration graph, one may wish to ask what impact gender has on the structure of the network.  Nevertheless, this analysis is often vulnerable to confounding variables. While gender very well may impact the structure of the graph, there is often a better explanatory variable for the structure.

There is space to bridge the gap between these two approaches. Rather than determining whether a feature impacts the graphical structure directly, we can use the community concept as a stepping stone in our analysis. Therefore, we extract which features have the largest impact on overall graphical structure. This can be thought of as a form of dimensionality-reduction where we only keep the node features that are useful in predicting community memberships.