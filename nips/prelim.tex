\section{Preliminaries}

We first need a model for community-like structure in a network. For this we adopt the stochastic block model (SBM) - widely used across academia. The premise is that each node in the graph belongs to a unique community called a block. The probability that two nodes are connected depends only on the block memberships of each. Graphs drawn from the SBM ensemble exhibit community structure. Specifically, we will use the microcanonical variant of the SBM, proposed by \citet{Peixoto-Bayesian-Microcanonical}. To allow for degree-variability between members of the same block, we must choose the degree-corrected formulation of the SBM (DC-SBM).

\begin{definition}[Microcanonical DC-SBM]
	\label{defn:microcan-dc-sbm}
	Let $N \in \Integers^{+}$ denote the number of vertices in an undirected graph. The block memberships are encoded by a vector $b \in [B]^N$,\footnote{We introduce the notation $[K] \coloneqq \{1, 2 \dots K\}$ to compactly define a set of $K$ indices. Clearly, $[K]$ is only defined for $K \in \Integers^+$.}
	where $B \in \Integers^{+}$ is the number of non-empty blocks.
	Let $e \in (\Integers_0^+)^{N \times N}$ be the matrix of edge counts between blocks. $e_{rs}$ is then the number of edges from block $r$ onto block $s$ -- or twice that number if $r=s$. For undirected graphs, $e$ is symmetric.
	Let $k \in (\Integers^+_0)^N$ be a vector denoting the degree sequence of the graph. $k_i$ is then the degree of vertex $i$.
	
	For the degree-corrected stochastic block model (DC-SBM), we say that the graph's adjacency matrix $A \in \{0,1\}^{N \times N}$ is generated as follows:
	%
	\begin{equation}
		A \sim \textrm{DC-SBM}_{\textrm{MC}} (b, e, k)
	\end{equation}
	%
	Where edges are placed uniformly at random but respecting the constraints imposed by $e$, $b$ and $k$ -- hence why this formulation is given the microcanonical moniker. Specifically, $A$ must satisfy the following:
	%
	\begin{equation}
		e_{rs} = \sum_{i, j \in [N]} A_{ij} \one \{b_i = r\} \one \{b_j = s\} \quad \forall r,s \in [B]
		\quad \textrm{and} \quad
		k_i = \sum_{j \in [N]} A_{ij} \quad \forall i \in [N]
		\label{eqn:sbm-constraints}
	\end{equation}
	%
	The additional parameters $N$ and $B$ are omitted as they can be inferred from the shapes of $b$ and $e$.
\end{definition}