\section{Experiments}
\label{sec:experiments}

We apply the developed methods to a variety of datasets. Datasets are chosen to span a range of node counts $N$, edge counts $E$ and feature space dimension $D$. We consider the following datasets:\footnote{For information as to how these data were collected and how personally identifiable information was removed, please refer to the cited papers.}

\begin{itemize}
	\item \textbf{Political books} \cite{polbooks} ($N=105, E=441, D=3$) -- network of Amazon book sales about U.S. politics, published close to the presidential election in 2004. Two books are connected if they were frequently co-purchased by the same customer. Vertex features encode the political affiliation of the author (liberal, conservative or neutral).
		
	\item \textbf{Primary school dynamic contacts} \cite{schools} ($N=238, E=5539, D=13$) -- network of face-to-face contacts amongst students and teachers at a primary school in Lyon, France. Two nodes are connected if the two parties shared a face-to-face interaction over the course of the day. Vertex features include class membership (one of 10 values 1A-5B), gender and whether or not the individual is a teacher or pupil. These data were collected on consecutive days in October 2009. We choose to analyse just the second day.
	
	\item \textbf{Facebook egonet} \cite{fb-snap} ($N=747, E=30025, D=480$) -- an assortment of Facebook egonets. These are networks of a particular user's friends list and all the connections within that. Vertex features are extracted from each user's profile and are fully anonymized. They include information about education history, languages spoken, gender, home-town, date of birth, name to give a few examples. We focus on the egonet with id 1912.

\end{itemize}

\begin{figure}[!h]
	\centering
	\begin{subfigure}[t]{0.28\linewidth}
		\centering
		\includegraphics[width=\linewidth]{polbooks-graph.png}
		\caption{Polbooks}
		\label{fig:polbooks-graph}
	\end{subfigure}
	\hfill
	\begin{subfigure}[t]{0.28\linewidth}
		\centering
		\includegraphics[width=\linewidth]{school-graph.png}
		\caption{School}
		\label{fig:school-graph}
	\end{subfigure}
	\hfill
	\begin{subfigure}[t]{0.28\linewidth}
		\centering
		\includegraphics[width=\linewidth]{fb-graph.png}
		\caption{Facebook egonet}
		\label{fig:fb-graph}
	\end{subfigure}
	\begin{subfigure}[t]{0.10\linewidth}
		\centering
		\includegraphics[width=0.8\linewidth]{10-vertical-legend.png}
		\caption{Legend}
		\label{fig:10-legend}
	\end{subfigure}
	\caption{Networks laid out and coloured according by inferred block memberships $\hat{y}$ for a given experiment iteration. Visualisation performed using \textit{graph-tool} \cite{peixoto_graph-tool_2014}}
	\label{fig:graphs-all}
\end{figure}

We require metrics to assess performance. This can be split into two separate components: the microcanonical SBM fit (concerned with the $b$-samples) and the fit of the feature-to-block generator (concerned with the $\theta$-samples). Starting with the SBM, $S(b)$ (equation \ref{eqn:dl-form}) can be interpreted as the description length of the partition imposed by $b$. It is only natural to divide this quantity by the number of entities (nodes and edges) in our graph $N+E$ to allow for rough comparison between graphs. This defines a simple metric to gauge the fit of the SBM: the description length per entity averaged over the $b$-samples (equation \ref{eqn:mean-dl}):
%
\begin{equation}
	\bar{S}_e \coloneqq \frac{1}{(N+E) |\Tcal_b|} \sum_{t\in \Tcal_b} S \left( b^{(t)} \right)
	\label{eqn:mean-dl}
\end{equation}
%
However, to assess the performance of the feature-to-block predictor, we must partition the vertex set $[N]$ into training and test sets: $\Gcal_0$ and $\Gcal_1$ respectively.\footnote{We choose to randomly create the partition on each experiment run such that a constant fraction $f$ of the available vertices go to form our training set $\Gcal_0$ and the remaining vertices are held out to form our test set $\Gcal_1$. For all the experiments $f=0.7$.}
The $b$-chain is run using the whole network but we only use vertices $v \in \Gcal_0$ to train the $\theta$-chain. As $|\Gcal_0| \neq |\Gcal_1|$ sets, we cannot use the un-normalised log target $U$ (equation \ref{eqn:U-form}) for comparison; the total cross-entropy loss is scaled by the size of each set but the prior term stays constant. We therefore must use the average cross-entropy loss over each set (equation \ref{eqn:cross-entropy-loss}):
%
\begin{equation}
	\bar{\Lcal}_\star \coloneqq \frac{1}{|\Tcal_\theta|} \sum_{t \in \Tcal_\theta} \Lcal_\star \left( \theta^{(t)} \right)
	\qquad \textrm{where} \qquad
	\Lcal_\star \left( \theta^{(t)} \right) \coloneqq \frac{1}{|\Gcal_\star|} \sum_{i \in \Gcal_\star}\sum_{j \in [B]} \hat{y}_{ij} \log \frac{1}{\phi_j \left(x_i; \theta^{(t)} \right)}
	\label{eqn:cross-entropy-loss}
\end{equation}
%
Where $\star \in \{0, 1\}$ has been introduced to toggle between training and test sets. Table \ref{tab:results} summarises the results for each experiment.\footnote{For a comprehensive list of the hyper-parameters used for each experiment please see appendix \ref{appdx:hyperparams}}

We also apply the dimensionality reduction method on the two higher dimensional datasets (the school and FB egonet). For this we leverage equation \ref{eqn:c-star}, to reduce the dimension from $D$ to $D'$ with $k=1$ to yield $c^*$. We then retrain the feature-block predictor using just the retained feature set $\Dcal'$ and report the loss over the training and test sets for the reduced classifier -- denoted $\bar{\Lcal}_0'$ and $\bar{\Lcal}_1'$ respectively. These values are also included in table \ref{tab:results}.

\begin{table}[!h]
	\centering
	\caption{Experimental results averaged over $n=10$ iterations (mean $\pm$ standard deviation)}
	\label{tab:results}
	\resizebox{\textwidth}{!}{%
	\begin{tabular}{c|ccc|c|cc|ccc}
		Dataset  & $B$ & $D$ & $D'$ & $\bar{S}_e$ & $\bar{\Lcal}_0$ & $\bar{\Lcal}_1$ & $c^*$ & $\bar{\Lcal}_0'$ & $\bar{\Lcal}_1'$  \\ \hline
		Polbooks & 3 & 3 & -- & $2.250 \pm 0.001$ & $0.584 \pm 0.033$ & $0.557 \pm 0.061$ & -- & -- & -- \\
		School & 10 & 13 & 10 & $1.894 \pm 0.006$ & $0.793 \pm 0.096$ & $0.914 \pm 0.112$ & $1.112 \pm 0.230$ & $0.791 \pm 0.096$ & $0.864 \pm 0.139$ \\
		FB egonet & 10  & 480 & 10 & $1.626 \pm 0.003$ & $1.305 \pm 0.034$ & $1.539 \pm 0.087$ & $0.942 \pm 0.042$ & $1.496 \pm 0.093$ & $1.578 \pm 0.104$
	\end{tabular}
	}
\end{table}

Table \ref{tab:results} already highlights some general trends in the results. Firstly, the variance of the test loss $\bar{\Lcal}_1$ tends to be higher than the training loss $\bar{\Lcal}_0$. This is expected as our test set is smaller than the training set and so more susceptible to variability in its construction. Indeed, much of the variance in the evaluation of $\bar{\Lcal}_0$ and $\bar{\Lcal}_1$ comes from the random partitioning of the graph into training and test sets $\Gcal_0$ and $\Gcal_1$. Secondly, it can be seen that the dimensionality reduction procedure brings the training and test losses closer together. This implies that the features we keep are indeed correlated with the underlying graphical partition.

The average description length per entity of the graph $\bar{S}_e$ has very low variance implying the detected communities can be found reliably (to within an arbitrary relabelling of blocks). For reference we plot an inferred partition for each of the graphs on figure \ref{fig:graphs-all}. The polbooks graph yields the cleanest separation between blocks but nonetheless the inferred partitions for the other datasets do succeed at partitioning the graph into densely connected clusters.

On figure \ref{fig:null-all} we give the sampled feature weights for the reduced feature-set $\Dcal'$ for a given experiment iteration (or just the original $\Dcal$ for the polbooks dataset). We go on to analyse each of these in turn.

\begin{figure}[!h]
	\centering
	\begin{subfigure}[t]{0.32\linewidth}
		\centering
		\vskip 0pt
		\includegraphics[width=\linewidth]{polbooks-null.png}
		\caption{Polbooks}
		\label{fig:polbooks-null}
	\end{subfigure}
	\hfill
	\begin{subfigure}[t]{0.32\linewidth}
		\centering
		\vskip 0pt
		\includegraphics[width=\linewidth]{school-null.png}
		\caption{School}
		\label{fig:school-null}
	\end{subfigure}
	\hfill
	\begin{subfigure}[t]{0.32\linewidth}
		\centering
		\vskip 0pt
		\includegraphics[width=\linewidth]{fb-null.png}
		\caption{Facebook egonet}
		\label{fig:fb-null}
	\end{subfigure}
	\caption{Reduced dimension feature-to-block generator weight samples}
	\label{fig:null-all}
\end{figure}

\subsection{Political books}

We wish to determine whether political affiliation is a good predictor of the overall network structure. We choose to partition the network into $B=3$ communities as we only have this many distinct values for political affiliation (conservative, liberal or neutral).

From, figure \ref{fig:polbooks-null}, we see that all 3 blocks have a distinct political affiliation as their largest positive component. This is strong evidence that political affiliation is indeed the axis which best predicts the 3-way natural partition of the graph into blocks.

From figure \ref{fig:polbooks-graph}, we note that there are very few edges between blocks 2 and 3 (the detected conservative and liberal blocks as per figure \ref{fig:school-null}). Block 1 effectively acts as a separator between blocks 2 and 3; very few readers venture all the way across the political aisle for their next book.

From table \ref{tab:results} we see that the training and test losses are very similar (the test loss mean is even below the training loss) and both are low in magnitude. This is very strong evidence that the features we have available (the author's political affiliation) are indeed the drivers of the high-level graphical structure at $B=3$.

\subsection{Primary school dynamic contacts}

We must first choose the number of blocks $B$ to define the coarseness of our analysis. A total of 10 school-classes would suggest that $B=10$ is a natural starting point. We visualise the inferred block memberships in figure \ref{fig:school-graph}.

As before, we sample the block-generator parameters $\theta$ and employ the dimensionality reduction technique with standard deviation multiplier $k=1$ to pick out the top $D'=10$ features. We then plot the weights for the surviving features $d \in \Dcal'$ on figure \ref{fig:school-null}. Immediately, we see that only the pupils' class memberships have survived (1A-5B); gender and teacher/student status have been discarded meaning that these are not good predictors of overall macro-structure.

The vast majority of blocks are composed of a single class. However, some blocks have 2 comparably good classes as their predictor. For example, block 2 contains classes 3A and 3B as its 2 best predictors. This suggests that the social divide between classes is less pronounced for pupils in year 3. Conversely, some classes are found to extend over two detected blocks (class 2B spans blocks 6 and 7) but we nonetheless do not have a feature which explains the division. The most surprising block is number 5 - which has comparable weightings for classes 5A and 1B. Perhaps there was a joint event between those two classes on the day the data were collected.

As for the training and test losses of the original and reduced classifiers. We see that the training loss stays roughly constant between the original and reduced classifiers $\bar{\Lcal}_0 \approx \bar{\Lcal}_0'$ but the test loss of the reduced classifier is ever-so-slightly decreased (within margin of error). This improved generalisation implies that the features we have discarded (gender and teacher/student status) do not impact the high-level graphical structure and simply lead to over-fitting in the original classifier. The features we have available do not capture the entire picture but they do quite well (modestly low overall loss).

\subsection{Facebook egonet}

This dataset was chosen to showcase the power of the dimensionality reduction technique as the feature-space has high dimension ($D=480$). We sample the $b$-chain, specifying $B=10$ total blocks and use this to construct the $\theta$-samples as before. 

We again apply the dimensionality reduction technique with $k=1$ to target $D'=10$ features in the reduced set. The remaining features (figure \ref{fig:fb-null}) are those that best explain the high-level community structure. The majority of the surviving features are education related. Nevertheless, for $D'=10$ we only have good explanations for the makeup of some of the detected blocks.

When the feature dimension is very large, it becomes increasingly likely that a particular feature may uniquely identify a small set of nodes. If these nodes are all part of the same community then the classifier will overfit for that particular parameter. The regularisation term imposed by the prior goes some way to alleviating this problem. Nevertheless, we see in figure \ref{fig:fb-null} that the feature \verb*|birthday-5| has a very high weight as it relates to block 1. It might be possible to shift the feature values such that they take values in $\{-1, 1\}$ rather than $\{0, 1\}$ but to accept more than one feature-group per block.

The training and test losses (table \ref{tab:results}) are the highest of the considered networks but nonetheless still low. This suggests that the features we have available only partly explain the high-level structure. Promisingly, after reducing the problem dimension our test loss barely changes, suggesting that the features we do keep are the best from the ones we have available at explaining the overall network structure. 

