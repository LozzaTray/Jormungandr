\section{Experiments}

We apply the developed methods to a variety of datasets, to show the power of the method across different scenarios. We choose datasets to span a range of node counts $N$, edge counts $E$ and feature space dimension $D$.

\begin{itemize}
	\item \textbf{Political books} \cite{polbooks} ($N=105, E=441, D=1$) - network of Amazon book sales about U.S. politics, published close to the presidential election in 2004. Two books are connected if they were frequently co-purchased by the same customer. Vertex features encode the broad political affiliation of the author (liberal, conservative, neutral).
		
	\item \textbf{Primary school dynamic contacts} \cite{schools} ($N=238, E=5539, D=13$) - network of face-to-face contacts amongst students and teachers at a primary school in Lyon, France. Two nodes are connected if the two parties shared a face-to-face interaction over the course of the day. Vertex features include class membership, gender and whether or not the student is a teacher or pupil. These data were collected on consecutive days in October 2009. We choose to analyse just the second day.
	
	\item \textbf{Facebook egonet} \cite{fb-snap} ($N=747, E=30025, D=480$) - an assortment of Facebook egonets. These are networks of a particular user's friends list and all the connections within that. Vertex features are extracted from each user's profile and are fully anonymized. We focus on the egonet with id 1912.
	
%	\item \textbf{Maier Facebook Egonet} \cite{FB-Maier}  ($N=349, E=2336, D=32$) - egonet of the author's Facebook friends list. Each vertex has been manually labelled with a variety of features describing their relationship to the author. For our purposes we remove all nodes of degree 1 (those that are only connected to the egonode) as these cannot be said to be part of any community present in the graph.
%		
%	\item \textbf{Law firm} \cite{LawFirm} - a network of relationships between members of a law firm. Each relationship is categorised according to type: coworkers, friends or advice.
%	
%	\item \textbf{Twitch users} \cite{twitch} - a network of user-user friendships on the streaming service Twitch. Vertex labels are extracted according to video-games played, location and streaming habits. This dataset is also broken down into disjoint networks according to language. We only consider the English users with is a subnet with $N=7126$ vertices and $E=35324$ edges).

\end{itemize}

We require metrics to assess the relative model evidence for the FFBM. This can be split into two separate components: the microcanonical SBM fit (concerned with the $b$-samples) and the fit of the feature-to-block generator (concerned with the $\theta$-samples). For each of these we can evaluate the mean of the unnormalised log-target of each Markov chain.
%
\begin{equation}
	\bar{S}_b \coloneqq \frac{1}{T_b} \sum_{t=1}^{T_b} S \left( b^{(t)} \right) \qquad \textrm{and} \qquad
	\bar{U}_\theta \coloneqq \frac{1}{T_\theta} \sum_{t=1}^{T_\theta} U \left( \theta^{(t)} \right)
\end{equation}
%
The lower these quantities, the better the fit of each stage of the model. To allow for rough comparison between datasets, we divide these quantities by the number of vertices in the graph $N$. Table \ref{tab:results} summarises the results for each experiment.

\begin{table}[!h]
	\centering
	\caption{FFBM fit for the datasets}
	\label{tab:results}
	\begin{tabular}{c|ccc|c|cc}
		Dataset & $N$ & $E$ & $D$ & $B$ & $\bar{S}_b /N$ & $\bar{U}_\theta /N$ \\ \hline
		Political books & 105  & 441 & 1 & 3 & 11.7 & 0.707 \\
		Primary school & 238 & 5539 & 13 &  18 & 43.0  & 2.75 \\
		FB egonet & 747 & 30025 & 480 & 10  &           &             \\
		       
	\end{tabular}
\end{table}


\subsection{Political books}

This dataset was collected by \citet{polbooks}. We wish to determine whether political affiliation is a good predictor of the overall network structure. We choose to partition the network into $B=3$ communities as we only have this many distinct values for political affiliation (conservative, liberal or neutral). The inferred block memberships are given in figure \ref{fig:books-graph}.

\begin{figure}[!h]
	\centering
	\begin{subfigure}{0.3\linewidth}
		\centering
		\includegraphics[width=\linewidth]{polbooks-graph.png}
		\fbox{\includegraphics[width=0.4\linewidth]{3-legend.png}}
		\caption{Graph coloured by inferred block memberships of each node}
		\label{fig:books-graph}
	\end{subfigure}
	\hfill
	\begin{subfigure}{0.5\linewidth}
		\centering
		\includegraphics[width=\linewidth]{polbooks-null.png}
		\caption{Feature-to-block generator sampled weight parameters for each block index}
		\label{fig:book-null}
	\end{subfigure}
	\caption{Political books network}
\end{figure}

\subsection{Primary school dynamic contacts}

These data were originally collected by \citet{schools} to quantify the transmission opportunities for respiratory infections within a primary school context (ages 6-11 in France). However, we seek to ask a simpler question. What features best describe how people interact with one another in a primary school context. The only vertex features we have available are school-class (one of 10 values - 2 per year group), gender and a distinction between teachers and pupils.

We must first choose the number of blocks to consider $B$ to define the coarseness of our analysis. A total of 10 classes would suggest that $B=10$ is a natural starting point. With this value of $B$ we first sample from the block membership posterior $b^{(t)} \sim p(b | A, X)$. We visualise the inferred partition in figure \ref{fig:school-graph}

\begin{figure}[!h]
	\centering
	\begin{subfigure}{0.45\linewidth}
		\centering
		\includegraphics[width=\linewidth]{school-graph.png}
		\fbox{\includegraphics[width=0.5\linewidth]{school-legend.png}}
		\caption{Graph coloured by inferred block memberships of each node}
		\label{fig:school-graph}
	\end{subfigure}
	\hfill
	\begin{subfigure}{0.45\linewidth}
		\centering
		\includegraphics[width=\linewidth]{school-null.png}
		\caption{Feature-to-block generator sampled weight parameters for each block index}
		\label{fig:school-null}
	\end{subfigure}
	\caption{Primary school dynamic contacts network}
\end{figure}


\subsection{Facebook egonet}

\begin{figure}[!h]
	\centering
	\begin{subfigure}{0.45\linewidth}
		\centering
		\includegraphics[width=\linewidth]{fb-graph.png}
		\fbox{\includegraphics[width=0.3\linewidth]{10-legend.png}}
		\caption{Graph coloured by inferred block memberships of each node}
		\label{fig:fb-graph}
	\end{subfigure}
	\hfill
	\begin{subfigure}{0.45\linewidth}
		\centering
		\includegraphics[width=\linewidth]{fb-null.png}
		\caption{Feature-to-block generator sampled weight parameters for each block index}
		\label{fig:fb-null}
	\end{subfigure}
	\caption{Primary school dynamic contacts network}
\end{figure}


