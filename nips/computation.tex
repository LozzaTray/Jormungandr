\section{Appendix: Implementation details}

\subsection{Algorithms}
\label{appdx:algorithms}

\begin{algorithm} % enter the algorithm environment
	\caption{Block membership sample generation} % give the algorithm a caption
	\label{alg:b-samples} % and a label for \ref{} commands later in the document
	\begin{algorithmic} % enter the algorithmic environment
		\Procedure{SampleBlockMemberships}{$A, T_b$}
		\State $b^{(0)} \gets \argmin_b S(b|A)$ \Comment{Implemented as greedy heuristic in \textit{graph-tool} library}
		\For{$t \in \{0, 1 \dots T_b - 1\}$}
		\State $b' \gets \sim q_b(b^{(t)}, b' | A)$
		\State $\log \alpha_b \gets \log \alpha_b(b^{(t)}, b' | A)$
		\State $\eta \gets \sim \textrm{Unif}(0,1)$
		\If{$\log \eta < \log \alpha_b$}
		\State $b^{(t+1)} \gets b'$
		\Else
		\State $b^{(t+1)} \gets b^{(t)}$
		\EndIf
		\EndFor
		
		\State \textbf{return} $\{b^{(t)}\}_{t=1}^{T_b}$
		\EndProcedure
	\end{algorithmic}
\end{algorithm}

\begin{algorithm} % enter the algorithm environment
	\caption{FFBM parameter pseudo-marginal inference} % give the algorithm a caption
	\label{alg:theta-samples} % and a label for \ref{} commands later in the document
	\begin{algorithmic} % enter the algorithmic environment
		\Procedure{SampleFeatureWeights}{$X, \{b^{(t)}\}, \Tcal_b, \sigma_\theta, s$}
		\State $\hat{Y}_{ij} \gets \frac{1}{|\Tcal_b|} \sum_{t \in \Tcal_b} \one \{ b^{(t)}_i = j\} \quad \forall i,j$
		\State $\theta^{(0)} \gets \sim \Gaussian(0, \sigma_\theta I)$
		
		\item[]
		
		\For{$t \in \{0, 1 \dots T_\theta - 1\}$}
		\State $\xi \gets \sim \Gaussian(0, I)$
		\State $h_t \gets \frac{s}{N} \cdot 250(1000 + t)^{-0.8}$
		\State $g_t \gets \nabla U(\theta^{(t)}| X, \hat{Y})$
		\item[]
		\State $\theta' \gets \theta^{(t)} - h_t \cdot g_t + \sqrt{2h_t} \cdot \xi$
		\State $\log \alpha_\theta \gets \log \alpha_\theta(\theta^{(t)}, \theta' | A, \hat{Y})$
		\State $\eta \gets \sim \textrm{Unif}(0,1)$
		\If{$\log \eta < \log \alpha_\theta$}
		\State $\theta^{(t+1)} \gets \theta'$
		\Else
		\State $\theta^{(t+1)} \gets \theta^{(t)}$
		\EndIf
		\EndFor
		
		\State \textbf{return} $\{\theta^{(t)}\}_{t=1}^{T_\theta}$
		\EndProcedure
	\end{algorithmic}
\end{algorithm}

\begin{algorithm} % enter the algorithm environment
	\caption{Dimensionality reduction} % give the algorithm a caption
	\label{alg:retained-set} % and a label for \ref{} commands later in the document
	\begin{algorithmic} % enter the algorithmic environment
		\Procedure{ReduceDimension}{$\{W^{(t)}\}, \Tcal_\theta, k, D'$}
		
		\State $(B, D) \gets W^{(0)}$.shape
		\State $\hat{\mu}_{ij} \gets \frac{1}{|\Tcal_\theta|} \sum_{t \in \Tcal_\theta} W_{ij}^{(t)} \quad \forall i \in [B], j \in [D]$
		\State $\hat{\sigma}_{ij} \gets \frac{1}{|\Tcal_\theta|} \sum_{t \in \Tcal_\theta} \left( W_{ij}^{(t)} - \hat{\mu}_{ij} \right)^2 \quad \forall i \in [B], j \in [D]$
		
		\item[]
		
		\For{$d \in [D]$}
			\For{$i \in [B]$}
				\State $l_i \gets \hat{\mu}_{id} - k \cdot \hat{\sigma}_{id}$
				\State $u_i \gets \hat{\mu}_{id} + k \cdot \hat{\sigma}_{id}$
				\If{$l_i \leq 0$ and $u_i \geq 0$}
					\State $l_i, u_i \gets 0$
				\EndIf
			\EndFor
			\State $c_d \gets \max_i \min\left(|l_i|, |u_i| \right)$
		\EndFor
		
		\item[]
		
		\State indexArray $\gets$ indexSort($c$, descending=True)[$0:D'$]
		\State $d^* \gets$ indexArray[-1]
		\State $\Dcal' \gets $ Set(indexArray)
		\State $c^* \gets c_{d^*}$
		\State \textbf{return} $\Dcal', c^*$
		\EndProcedure
	\end{algorithmic}
\end{algorithm}

\FloatBarrier
\subsection{Hyperparameter values}
\label{appdx:hyperparams}

\begin{table}[!h]
	\centering
	\caption{Hyper-parameter values for each experiment}
	\label{tab:hyperparams}
	\resizebox{\textwidth}{!}{%
		\begin{tabular}{c|ccc|ccc|cccc|cc|cccc}
			Dataset & 
			$B$ & $f$ & $\sigma_\theta$ & 
			$T_b$ & $\kappa_b$ & $\lambda_b$ & 
			$T_\theta$ & $\kappa_\theta$ & $\lambda_\theta$ & $s$ &
			$k$ & $D'$ &
			$T_\theta'$ & $\kappa_\theta'$ & $\lambda_\theta'$ & $s'$
			\\ \hline
			Polbooks &
			3 & 0.7 & 1 &
			1,000 & 0.2 & 5 &
			10,000 & 0.4 & 10 & 0.05 &
			-- & -- & 
			-- & -- & -- & -- \\
			School &
			10 & 0.7 & 1 &
			1,000 & 0.2 & 5 &
			10,000 & 0.4 & 10 & 0.2 &
			1 & 10 & 
			10,000 & 0.4 & 10 & 0.2 \\
			FB Egonet &
			10 & 0.7 & 1 &
			1,000 & 0.2 & 5 &
			10,000 & 0.4 & 10 & 0.017 &
			1 & 10 & 
			10,000 & 0.4 & 10 & 0.5 \\
		\end{tabular}
	}
\end{table}

\subsection{Hardware specification}
\label{appdx:imp-details}

All data analysis and visualisation was implemented in Python. Full source code is available in the supplementary material. The scripts were run using a standard PC using the Windows Subsystem for Linux (WSL) environment. Specs are:

\begin{itemize}
	\item \textbf{CPU}: Intel(R) Core(TM) i7-1065G7
	\item \textbf{RAM}: 8GB
	\item \textbf{GPU}: Intel(R) Iris(R) Plus Graphics
\end{itemize}

On this hardware each experiment iteration took the following amount of time to execute:

\begin{table}[!h]
	\centering
	\caption{Compute-time for each experiment}
	\label{tab:compute-time}
	\begin{tabular}{c|ccc|c}
		Dataset & $b$-chain & $\theta$-chain & Reduced $\theta$-chain & Overall compute time \\ \hline
		Polbooks & $\sim$1s & $\sim$10s & -- & $\sim$11s \\
		School & $\sim$1s & $\sim$7s & $\sim$7s & $\sim$15s \\
		FB Egonet & $\sim$2s & $\sim$50s & $\sim$8s & $\sim$60s
	\end{tabular}
\end{table}

