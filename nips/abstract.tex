\begin{abstract}
	Labelled networks are an extremely common and important form of data. A typical inference goal is to determine how the vertex labels (called features) affect graphical structure. The standard approach to this problem is to partition the network into blocks grouped by distinct values of the feature of interest. We then use a block-based random graph model - typically a variant of the stochastic block model (SBM) - to test for evidence these extracted feature-based communities interact differently with one another.
	
	Nevertheless, these feature-based communities are often not a natural partition of the graph and thus the model is not a good fit. With this in mind, we present a novel generative model, which we call the feature-first block model (FFBM), for better describing vertex-labelled undirected graphs. We present a method to efficiently sample the FFBM parameters. This allows us to automatically determine which features best explain the block-based graphical structure. Importantly, this analysis can be performed using the whole feature-set rather than considering each feature independently. 
	
	We apply the developed method to a variety of network data to extract the most important features along which the vertices cluster. Those that do not impact the high-level structure can be discarded to reduce the problem dimension. In the case the vertex features available do not readily explain the community structure in the resulting network, the approach detects this and is protected against over-confidence. Future work may benefit from extending the FFBM to multiple hierarchical levels. This allows the structure to be explained at each level of coarseness. 
\end{abstract}