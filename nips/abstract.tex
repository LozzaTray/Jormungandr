\begin{abstract}
	Labelled networks are an extremely common and important form of data. A typical inference goal is to determine how the vertex labels (called features) affect graphical structure. The standard approach to this problem has been to partition the network into blocks grouped by distinct values of the feature of interest. A block-based random graph model - typically a variant of the stochastic block model (SBM) - is then used to test for evidence of asymmetric behaviour within these feature-based communities.
	
	Nevertheless, these feature-based communities are often not a natural partition of the graph and thus the models employed are rarely a good fit. With this in mind, we present a novel generative model, which we call the feature-first block model (FFBM), for better describing vertex-labelled undirected graphs. This allows us to perform richer queries on labelled networks. We present a method to efficiently sample the FFBM parameters for inference. The FFBM's structure is kept deliberately simple to retain easy interpretability of the parameter values.
	
	We apply the developed methods to a variety of network data to extract the most important features along which the vertices divide themselves. The greatest advantage of the proposed approach is that the whole feature-space is used automatically and features can be rank-ordered implicitly according to impact. Any features that do not greatly impact the high-level structure can be discarded to reduce the problem dimension. In the case the vertex features available do not readily explain the community structure in the resulting network, the approach detects this and is protected against over-confidence. Future work may benefit from extending the FFBM to multiple hierarchical levels. This would allow the structure to be explained at each level of coarseness rather than simply at the highest level. 
\end{abstract}