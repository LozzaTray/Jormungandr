\documentclass[]{article}

%opening
\title{Jormungandr - Reading Notes}
\author{Lawrence Tray \\ Ioannis Kontoyiannis}

%packages
\usepackage[margin=0.5in]{geometry}
\usepackage{graphicx}
\usepackage{amsmath}
\usepackage{amssymb}
\usepackage{hyperref}
\usepackage{caption}
\usepackage{subcaption}
\usepackage{mathtools}
\usepackage{parskip}

\usepackage[backend=bibtex]{biblatex}
\addbibresource{sources.bib}

\begin{document}

\maketitle

\section{Initial Reading}

Summary of \cite{Abbe}.

Types of recovery:
\begin{enumerate}
	\item Exact
	\item Almost Exact
	\item Partial
	\item Weak
\end{enumerate}

For 2 balanced communities in a SSBM (Symmetric SBM) once can apply linear algebra pproaches to solve efficiently. But for more communities, the state of the art is belief propagation. 

Many properties are given as functions of the number of edges $n$. Therefore can maybe sub-partition graph to see how properties scale with $n$ to fit the most appropriate model.

Best algorithm for Weak recovery is Belief Propagation (linear version in \cite{Linear-ABP}).

Also in \cite{SBM-Achieve}.

\begin{equation}
	\mathcal{P} (||\Omega_i \cap S| / |)
\end{equation}

For graph layout, NX uses spring layout. This is the Fruchterman-Reingold force-directed algorithm.

\nocite{*}
\printbibliography


\end{document}
